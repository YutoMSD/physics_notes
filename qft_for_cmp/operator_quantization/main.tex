\documentclass{report}
\usepackage{luatexja}
\usepackage{amsmath, amssymb, type1cm, amsfonts, latexsym, mathtools, bm, amsthm, url, color}
\usepackage{multirow, hyperref, longtable, dcolumn, tablefootnote, physics}
\usepackage{tabularx, footmisc, colortbl, here, usebib, microtype}
\usepackage{graphicx, luacode, caption, fancyhdr}
\usepackage[top = 20truemm, bottom = 20truemm, left = 20truemm, right = 20truemm]{geometry}
\usepackage{ascmac, fancybox, color, tabularray, subcaption}
\usepackage{luatexja-fontspec, multicol}
\usepackage{upgreek, colortbl, mhchem}
\usepackage{biblatex, array, truthtable}
\usepackage{listings, jvlisting}
\usepackage{xcolor, siunitx, float, dcolumn}
\sisetup{
  table-format = 1.5, % 小数点以下の桁数を指定
  table-number-alignment = center, % 数値の中央揃え
}
% \abovedisplayskip = 0pt
% \belowdisplayskip = 0pt
\allowdisplaybreaks
% \DeclarePairedDelimiter{\abs}{\lvert}{\rvert}
\newcolumntype{t}{!{\vrule width 0.1pt}}
\newcolumntype{b}{!{\vrule width 1.5pt}}
\UseTblrLibrary{amsmath, booktabs, counter, diagbox, functional, hook, html, nameref, siunitx, varwidth, zref}
\setlength{\columnseprule}{0.4pt}
\captionsetup[figure]{font = bf}
\captionsetup[table]{font = bf}
\captionsetup[lstlisting]{font = bf}
\captionsetup[subfigure]{font = bf, labelformat = simple}
\setcounter{secnumdepth}{5}
\newcolumntype{d}{D{.}{.}{5}}
\newcolumntype{M}[1]{>{\centering\arraybackslash}m{#1}}
\everymath{\displaystyle}
\DeclareMathOperator*{\AND}{\cdot}
\DeclareMathOperator*{\NAND}{NAND}
% \DeclareMathOperator*{\NOT}{NOT}
\DeclareMathOperator*{\OR}{+}
% \let\oldbar\bar
\renewcommand{\i}{\mathrm{i}}
\renewcommand{\laplacian}{\Delta}
\newcommand{\NOT}[1]{\overline{#1}}
\renewcommand{\hat}[1]{\overhat{#1}}
\renewcommand{\thesubfigure}{(\alph{subfigure})}
\newcommand{\m}[3]{\multicolumn{#1}{#2}{#3}}
\renewcommand{\r}[1]{\mathrm{#1}}
\newcommand{\e}{\mathrm{e}}
\newcommand{\Ef}{E_{\mathrm{F}}}
\renewcommand{\c}{\si{\degreeCelsius}}
\renewcommand{\d}{\r{d}}
\renewcommand{\t}[1]{\texttt{#1}}
\newcommand{\kb}{k_{\mathrm{B}}}
\renewcommand{\phi}{\varphi}
% \newcommand{\dv}[3]{\frac{\d #1}{\d #2}}
% \newcommand{\pdv}[2]{\frac{\partial #1}{\partial #2}}
% \newcommand{\qtys}[#1]{\left(#1 \right)}
% \newcommand{\qtym}[#1]{\left\{#1\right\}}
% \newcommand{\qtyl}[#1]{\left[#1\right]}
\newcommand{\reff}[1]{\textbf{図\ref{#1}}}
\newcommand{\reft}[1]{\textbf{表\ref{#1}}}
\newcommand{\refe}[1]{\textbf{式\eqref{#1}}}
\newcommand{\refp}[1]{\textbf{コード\ref{#1}}}
\newcommand{\refa}[1]{\textbf{\ref{#1}}}
\renewcommand{\lstlistingname}{コード}
\renewcommand{\theequation}{\thesection.\arabic{equation}}
\renewcommand{\footrulewidth}{0.4pt}
\newcommand{\mar}[1]{\textcircled{\scriptsize #1}}
\newcommand{\combination}[2]{{}_{#1} \mathrm{C}_{#2}}
\newcommand{\thline}{\noalign{\hrule height 0.1pt}}
\newcommand{\bhline}{\noalign{\hrule height 1.5pt}}
\newcommand*{\myCurrentTime}{
  \directlua{ my_current_time() }
}
\newcommand{\Rnum}[1]{
  \ifnum #1 = 1
    I
  \fi
  \ifnum #1 = 2
    I\hspace{-1.2pt}I
  \fi
  \ifnum #1 = 3
    I\hspace{-1.2pt}I\hspace{-1.2pt}I
  \fi
  \ifnum #1 = 4
    I\hspace{-1.2pt}V
  \fi
  \ifnum #1 = 5
    V
  \fi
  \ifnum #1 = 6
    V\hspace{-1.2pt}I
  \fi
  \ifnum #1 = 7
    V\hspace{-1.2pt}I\hspace{-1.2pt}I
  \fi
  \ifnum #1 = 8
    V\hspace{-1.2pt}I\hspace{-1.2pt}I\hspace{-1.2pt}I
  \fi
  \ifnum #1 = 9
    I\hspace{-1.2pt}X
  \fi
  \ifnum #1 = 10
    X
  \fi
}
\newcommand{\cover}{
  \renewcommand{\arraystretch}{3}
  \title{物理情報工学実験報告書}
  \date{}
  \author{}
  \maketitle
  \begin{table}[H]
    \begin{flushright}
      2024年度
    \end{flushright}
    \begin{center}
      \begin{tabularx}{150mm}{|>{\centering}p{40mm}|>{\centering}p{25mm}|>{\centering}p{30mm}|>{\centering\arraybackslash}X|}
        \hline
        \Large{実験テーマ} & \multicolumn{3}{c|}{\Large{A1(直流安定化電源)}} \\ \hline
        \Large{担当教員名} & \multicolumn{3}{c|}{\Large{塚田孝祐}} \\ \hline
        \Large{実験整理番号} & \Large{002} & \Large{実験者氏名} & \Large{青木\ 陽}\\ \hline
        \Large{共同実験者氏名} & \multicolumn{3}{c|}{} \\ \hline
        \Large{曜日組} & \Large{火1班} & \Large{実験日} & \Large{6月25日} \\ \hline
        \Large{実験回} & \Large{9} & \Large{報告書提出日} & \Large{\myCurrentTime}\\ \hline
      \end{tabularx}
    \end{center}
  \end{table}
  \thispagestyle{empty} 
  \addtocounter{page}{-1}
  \clearpage
  \renewcommand{\arraystretch}{1.0}
}
\pagestyle{fancy}
\chead{物性物理II}
\rhead{}
\cfoot{\thepage}
\lhead{}
\rfoot{\t{harry\_arbrebleu}}
\setcounter{tocdepth}{4}
\makeatletter
\@addtoreset{equation}{subsection}
\makeatother
\begin{luacode*}
  function my_current_time()
    local date = os.date("*t")
    local year = date.year
    local month = date.month
    local day = date.day
    local hour = date.hour
    local min = date.min
    local sec = date.sec
    local formatted_date = string.format("%d月%d日", month, day)
    tex.sprint(formatted_date)
  end
\end{luacode*}
\lstset{
  language = Matlab, % Set the language for the code
  basicstyle = {\ttfamily},
  identifierstyle = {\small},
  commentstyle = \color{red},
  keywordstyle = \color{blue},
  ndkeywordstyle = {\small},
  stringstyle = \color{orange},
  frame={tb},
  breaklines = true,
  columns=[l]{fullflexible},
  xrightmargin = 5mm,
  xleftmargin = 5mm,
  numberstyle = {\ttfamily\scriptsize},
  stepnumber = 1,
  numbersep = 1mm,
  lineskip = -0.5ex,
  showstringspaces = false,
  numbers = left,
  frame = lines,
  backgroundcolor = \color{gray!10},
  rulecolor = \color{black!30},
}

\definecolor{mygray}{rgb}{0.5,0.5,0.5}
\definecolor{mymauve}{rgb}{0.58,0,0.82}
\definecolor{mygreen}{rgb}{0,0.6,0}

\lstset{ %
  backgroundcolor=\color{white},   % 背景色
  basicstyle=\ttfamily\footnotesize, % 基本の書体スタイル
  breakatwhitespace=false,        % 空白で行分割しない
  breaklines=true,                % 長い行は分割する
  captionpos=b,                   % キャプションの位置
  commentstyle=\color{mygreen},   % コメントのスタイル
  extendedchars=true,             % 非 ASCII 文字をサポート
  frame=single,                   % フレームの表示
  keywordstyle=\color{blue},      % キーワードのスタイル
  language=[LaTeX]TeX,            % 言語を LaTeX に設定
  numbers=left,                   % 行番号を左側に表示
  numbersep=5pt,                  % 行番号とコードの間の距離
  numberstyle=\tiny\color{mygray}, % 行番号のスタイル
  rulecolor=\color{black},        % 枠線の色
  showspaces=false,               % スペースを表示しない
  showstringspaces=false,         % 文字列内のスペースを表示しない
  showtabs=false,                 % タブを表示しない
  stepnumber=1,                   % 行番号を表示する間隔
  stringstyle=\color{mymauve},    % 文字列のスタイル
  tabsize=2,                      % タブの幅
  title=\lstname                  % タイトル
}
\lstset{
  language = C++, % Set the language for the code
  basicstyle = {\ttfamily},
  identifierstyle = {\small},
  commentstyle = \color{red},
  keywordstyle = \color{blue},
  ndkeywordstyle = {\small},
  stringstyle = \color{orange},
  frame={tb},
  breaklines = true,
  columns=[l]{fullflexible},
  xrightmargin = 5mm,
  xleftmargin = 5mm,
  numberstyle = {\ttfamily\scriptsize},
  stepnumber = 1,
  numbersep = 1mm,
  lineskip = -0.5ex,
  showstringspaces = false,
  numbers = left,
  frame = lines,
  backgroundcolor = \color{gray!10},
  rulecolor = \color{black!30},
}
\title{物性物理II}
\date{\today}
\author{\t{harry\_arbrebleu}}
\addbibresource{ref.bib}
\defbibheading{bunken}[\refname]{\section*{#1}}
\begin{document}
演算子がどのように第2量子化されるかを見る.

まずは1粒子に作用する演算子を足し合わせた
\begin{align}
  \hat{F} = \sum_i \hat{f}(\bm{r}_i)
\end{align}
が場の演算子を用いてどのように表されるのかを考える.$\hat{f}$の例は,運動エネルギー$\frac{\hat{p}_i^2}{2m}$や中心力ポテンシャル$V(\bm{r}_i)$などである.
位置$\bm{r}_i$固有状態$\lambda$にある粒子の波動関数を$\v_{\lambda}(\bm{r}_i)$と書くことにする.
これを使って多体系全体の状態$\ket{n}$を
\begin{align}
  \ket{n} = \ket{v_\alpha(\bm{r}_1)}\otimes\ket{v_\beta(\bm{r}_2)}\otimes\cdots\otimes\ket{v_\mu(\bm{r}_n)} = \ket{v_\alpha(\bm{r}_1),v_\beta(\bm{r}_2),\cdots,v_\nu(\bm{r}_n)}
\end{align}
と書く.まず,
\begin{align}
  \hat{f}(\bm{r}_i)\ket{v_\lambda(\bm{r}_i)} &= \sum_{\mu}\ket{v_\mu(\bm{r}_i)}\bra{v_\mu(\bm{r}_i)}\hat{f}(\bm{r}_i)\ket{v_\lambda(\bm{r}_i)}\\
  &= \sum_{\mu} \bra{v_\mu(\bm{r}_i)}\hat{f}(\bm{r}_i)\ket{v_\lambda(\bm{r}_i)}\ket{v_\mu(\bm{r}_i)}
\end{align}
が成り立つ.ここで,
\begin{align}
  \bra{v_\mu(\bm{r})}\hat{f}(\bm{r})\ket{v_\lambda(\bm{r})} = \int \dd[3]{\bm{r}}\v_{\mu}^*(\bm{r})\hat{f}(\bm{r})\v_{\lambda}(\bm{r})
\end{align}
である.
これを用いて
\begin{align}
  \hat{f}(\bm{r}_i)\ket{n} &= \hat{f}(\bm{r}_i)\ket{v_\alpha(\bm{r}_1),v_\beta(\bm{r}_2),\cdots,v_\lambda(\bm{r}_i),\cdots,v_\nu(\bm{r}_n)}\\
  &= \ket{v_\alpha(\bm{r}_1)}\otimes\ket{v_\beta(\bm{r}_2)}\otimes\cdots\qty(\hat{f}(\bm{r}_i)\ket{v_\lambda(\bm{r}_i)})\cdots\otimes\ket{v_\nu(\bm{r}_n)}\\
  &= \sum_{\mu} \bra{v_\mu(\bm{r}_i)}\hat{f}(\bm{r}_i)\ket{v_\lambda(\bm{r}_i)}\ket{v_\alpha(\bm{r}_1),v_\beta(\bm{r}_2),\cdots,v_\mu(\bm{r}_i),\cdots,v_\nu(\bm{r}_n)}\\
\end{align}
を得る.ここで,
\begin{align}
  \ket{v_\alpha(\bm{r}_1),v_\beta(\bm{r}_2),\cdots,v_\mu(\bm{r}_i),\cdots,v_\nu(\bm{r}_n)} = \hat{a}_\mu^{\dagger}\hat{a}_\lambda\ket{v_\alpha(\bm{r}_1),v_\beta(\bm{r}_2),\cdots,v_\lambda(\bm{r}_i),\cdots,v_\nu(\bm{r}_n)} =  \hat{a}_\mu^{\dagger}\hat{a}_\lambda\ket{n}
\end{align}
であるので,
\begin{align}
  \hat{F}\ket{n} &= \sum_i \hat{f}(\bm{r}_i)\ket{n}\\
  &= \sum_{i}\sum_{\mu} \bra{v_\mu(\bm{r}_i)}\hat{f}(\bm{r}_i)\ket{v_\lambda(\bm{r}_i)}\hat{a}_\mu^{\dagger}\hat{a}_\lambda\ket{n}\\
  &= \sum_{\mu,\lambda}\sum_{i} \bra{v_\mu(\bm{r})}\hat{f}(\bm{r})\ket{v_\lambda(\bm{r})}\hat{a}_\mu^{\dagger}\hat{a}_\lambda\ket{n}\\
\end{align}
となる.よって,1粒子に作用する演算子は第2量子化により
\begin{align}
  \hat{F} = \sum_{\lambda,\mu} \bra{v_\mu(\bm{r})}\hat{f}(\bm{r})\ket{v_\lambda(\bm{r})}\hat{a}_\mu^{\dagger}\hat{a}_\lambda \label{f}
\end{align}
と書ける.

2つの粒子の相互作用を表す演算子
\begin{align}
  \hat{G} = \sum_{i\neq j} \hat{g}(\bm{r}_i,\bm{r}_j)
\end{align}
の第2量子化も同様に
\begin{align}
  \hat{G} = \frac{1}{2} \sum_{\alpha,\beta,\lambda,\mu} \bra{v_\alpha(\bm{r}_1)v_\beta(\bm{r}_2)}\hat{g}(\bm{r}_1,\bm{r}_2)\ket{v_\lambda(\bm{r}_1)v_\mu(\bm{r}_2)}\hat{a}_\alpha^{\dagger}\hat{a}_\beta^{\dagger}\hat{a}_\mu\hat{a}_\lambda \label{g}
\end{align}
である.

さらに,場の演算子は座標の依存性のみを考えると,
\begin{align}
  \psi(\bm{r}) = \sum_i v_i(\bm{r})\hat{a}_i,\ \psi(\bm{r})^{\dagger} = \sum_i v_i^*(\bm{r})\hat{a}_i^{\dagger} \label{field-v}
\end{align}
と書けるのであった.これを用いると$\hat{F}$と$\hat{G}$は
\begin{align}
  \hat{F} &= \int \dd[3]{\bm{r}}\psi^{\dagger}(\bm{r})\hat{f}(\bm{r})\psi(\bm{r}) \label{f-field} \\
  \hat{G} &= \frac{1}{2}\int\dd[3]{\bm{r}_1}\int\dd[3]{\bm{r}_2}\psi^{\dagger}(\bm{r}_1)\psi^{\dagger}(\bm{r}_2)\hat{g}(\bm{r}_1,\bm{r}_2)\psi(\bm{r}_2)\psi(\bm{r}_1) \label{g-field}
\end{align}
と表される.確かに\refe{f-field}と\refe{g-field}に\refe{field-v}を代入すると,\refe{f}と\refe{g}に一致する.

\end{document}