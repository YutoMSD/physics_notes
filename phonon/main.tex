\documentclass{report}
\usepackage{luatexja}
\usepackage{amsmath, amssymb, type1cm, amsfonts, latexsym, mathtools, bm, amsthm, url, color}
\usepackage{multirow, hyperref, longtable, dcolumn, tablefootnote, physics}
\usepackage{tabularx, footmisc, colortbl, here, usebib, microtype}
\usepackage{graphicx, luacode, caption, fancyhdr}
\usepackage[top = 20truemm, bottom = 20truemm, left = 20truemm, right = 20truemm]{geometry}
\usepackage{ascmac, fancybox, color, tabularray, subcaption}
\usepackage{luatexja-fontspec, multicol}
\usepackage{upgreek, colortbl, mhchem}
\usepackage{biblatex, array, truthtable}
\usepackage{listings, jvlisting}
\usepackage{xcolor, siunitx, float, dcolumn}
\sisetup{
  table-format = 1.5, % 小数点以下の桁数を指定
  table-number-alignment = center, % 数値の中央揃え
}
% \abovedisplayskip = 0pt
% \belowdisplayskip = 0pt
\allowdisplaybreaks
% \DeclarePairedDelimiter{\abs}{\lvert}{\rvert}
\newcolumntype{t}{!{\vrule width 0.1pt}}
\newcolumntype{b}{!{\vrule width 1.5pt}}
\UseTblrLibrary{amsmath, booktabs, counter, diagbox, functional, hook, html, nameref, siunitx, varwidth, zref}
\setlength{\columnseprule}{0.4pt}
\captionsetup[figure]{font = bf}
\captionsetup[table]{font = bf}
\captionsetup[lstlisting]{font = bf}
\captionsetup[subfigure]{font = bf, labelformat = simple}
\setcounter{secnumdepth}{5}
\newcolumntype{d}{D{.}{.}{5}}
\newcolumntype{M}[1]{>{\centering\arraybackslash}m{#1}}
\everymath{\displaystyle}
\DeclareMathOperator*{\AND}{\cdot}
\DeclareMathOperator*{\NAND}{NAND}
% \DeclareMathOperator*{\NOT}{NOT}
\DeclareMathOperator*{\OR}{+}
% \let\oldbar\bar
\renewcommand{\i}{\mathrm{i}}
\renewcommand{\laplacian}{\Delta}
\newcommand{\NOT}[1]{\overline{#1}}
\renewcommand{\hat}[1]{\overhat{#1}}
\renewcommand{\thesubfigure}{(\alph{subfigure})}
\newcommand{\m}[3]{\multicolumn{#1}{#2}{#3}}
\renewcommand{\r}[1]{\mathrm{#1}}
\newcommand{\e}{\mathrm{e}}
\newcommand{\Ef}{E_{\mathrm{F}}}
\renewcommand{\c}{\si{\degreeCelsius}}
\renewcommand{\d}{\r{d}}
\renewcommand{\t}[1]{\texttt{#1}}
\newcommand{\kb}{k_{\mathrm{B}}}
\renewcommand{\phi}{\varphi}
% \newcommand{\dv}[3]{\frac{\d #1}{\d #2}}
% \newcommand{\pdv}[2]{\frac{\partial #1}{\partial #2}}
% \newcommand{\qtys}[#1]{\left(#1 \right)}
% \newcommand{\qtym}[#1]{\left\{#1\right\}}
% \newcommand{\qtyl}[#1]{\left[#1\right]}
\newcommand{\reff}[1]{\textbf{図\ref{#1}}}
\newcommand{\reft}[1]{\textbf{表\ref{#1}}}
\newcommand{\refe}[1]{\textbf{式\eqref{#1}}}
\newcommand{\refp}[1]{\textbf{コード\ref{#1}}}
\newcommand{\refa}[1]{\textbf{\ref{#1}}}
\renewcommand{\lstlistingname}{コード}
\renewcommand{\theequation}{\thesection.\arabic{equation}}
\renewcommand{\footrulewidth}{0.4pt}
\newcommand{\mar}[1]{\textcircled{\scriptsize #1}}
\newcommand{\combination}[2]{{}_{#1} \mathrm{C}_{#2}}
\newcommand{\thline}{\noalign{\hrule height 0.1pt}}
\newcommand{\bhline}{\noalign{\hrule height 1.5pt}}
\newcommand*{\myCurrentTime}{
  \directlua{ my_current_time() }
}
\newcommand{\Rnum}[1]{
  \ifnum #1 = 1
    I
  \fi
  \ifnum #1 = 2
    I\hspace{-1.2pt}I
  \fi
  \ifnum #1 = 3
    I\hspace{-1.2pt}I\hspace{-1.2pt}I
  \fi
  \ifnum #1 = 4
    I\hspace{-1.2pt}V
  \fi
  \ifnum #1 = 5
    V
  \fi
  \ifnum #1 = 6
    V\hspace{-1.2pt}I
  \fi
  \ifnum #1 = 7
    V\hspace{-1.2pt}I\hspace{-1.2pt}I
  \fi
  \ifnum #1 = 8
    V\hspace{-1.2pt}I\hspace{-1.2pt}I\hspace{-1.2pt}I
  \fi
  \ifnum #1 = 9
    I\hspace{-1.2pt}X
  \fi
  \ifnum #1 = 10
    X
  \fi
}
\newcommand{\cover}{
  \renewcommand{\arraystretch}{3}
  \title{物理情報工学実験報告書}
  \date{}
  \author{}
  \maketitle
  \begin{table}[H]
    \begin{flushright}
      2024年度
    \end{flushright}
    \begin{center}
      \begin{tabularx}{150mm}{|>{\centering}p{40mm}|>{\centering}p{25mm}|>{\centering}p{30mm}|>{\centering\arraybackslash}X|}
        \hline
        \Large{実験テーマ} & \multicolumn{3}{c|}{\Large{A1(直流安定化電源)}} \\ \hline
        \Large{担当教員名} & \multicolumn{3}{c|}{\Large{塚田孝祐}} \\ \hline
        \Large{実験整理番号} & \Large{002} & \Large{実験者氏名} & \Large{青木\ 陽}\\ \hline
        \Large{共同実験者氏名} & \multicolumn{3}{c|}{} \\ \hline
        \Large{曜日組} & \Large{火1班} & \Large{実験日} & \Large{6月25日} \\ \hline
        \Large{実験回} & \Large{9} & \Large{報告書提出日} & \Large{\myCurrentTime}\\ \hline
      \end{tabularx}
    \end{center}
  \end{table}
  \thispagestyle{empty} 
  \addtocounter{page}{-1}
  \clearpage
  \renewcommand{\arraystretch}{1.0}
}
\pagestyle{fancy}
\chead{物性物理II}
\rhead{}
\cfoot{\thepage}
\lhead{}
\rfoot{\t{harry\_arbrebleu}}
\setcounter{tocdepth}{4}
\makeatletter
\@addtoreset{equation}{subsection}
\makeatother
\begin{luacode*}
  function my_current_time()
    local date = os.date("*t")
    local year = date.year
    local month = date.month
    local day = date.day
    local hour = date.hour
    local min = date.min
    local sec = date.sec
    local formatted_date = string.format("%d月%d日", month, day)
    tex.sprint(formatted_date)
  end
\end{luacode*}
\lstset{
  language = Matlab, % Set the language for the code
  basicstyle = {\ttfamily},
  identifierstyle = {\small},
  commentstyle = \color{red},
  keywordstyle = \color{blue},
  ndkeywordstyle = {\small},
  stringstyle = \color{orange},
  frame={tb},
  breaklines = true,
  columns=[l]{fullflexible},
  xrightmargin = 5mm,
  xleftmargin = 5mm,
  numberstyle = {\ttfamily\scriptsize},
  stepnumber = 1,
  numbersep = 1mm,
  lineskip = -0.5ex,
  showstringspaces = false,
  numbers = left,
  frame = lines,
  backgroundcolor = \color{gray!10},
  rulecolor = \color{black!30},
}

\definecolor{mygray}{rgb}{0.5,0.5,0.5}
\definecolor{mymauve}{rgb}{0.58,0,0.82}
\definecolor{mygreen}{rgb}{0,0.6,0}

\lstset{ %
  backgroundcolor=\color{white},   % 背景色
  basicstyle=\ttfamily\footnotesize, % 基本の書体スタイル
  breakatwhitespace=false,        % 空白で行分割しない
  breaklines=true,                % 長い行は分割する
  captionpos=b,                   % キャプションの位置
  commentstyle=\color{mygreen},   % コメントのスタイル
  extendedchars=true,             % 非 ASCII 文字をサポート
  frame=single,                   % フレームの表示
  keywordstyle=\color{blue},      % キーワードのスタイル
  language=[LaTeX]TeX,            % 言語を LaTeX に設定
  numbers=left,                   % 行番号を左側に表示
  numbersep=5pt,                  % 行番号とコードの間の距離
  numberstyle=\tiny\color{mygray}, % 行番号のスタイル
  rulecolor=\color{black},        % 枠線の色
  showspaces=false,               % スペースを表示しない
  showstringspaces=false,         % 文字列内のスペースを表示しない
  showtabs=false,                 % タブを表示しない
  stepnumber=1,                   % 行番号を表示する間隔
  stringstyle=\color{mymauve},    % 文字列のスタイル
  tabsize=2,                      % タブの幅
  title=\lstname                  % タイトル
}
\lstset{
  language = C++, % Set the language for the code
  basicstyle = {\ttfamily},
  identifierstyle = {\small},
  commentstyle = \color{red},
  keywordstyle = \color{blue},
  ndkeywordstyle = {\small},
  stringstyle = \color{orange},
  frame={tb},
  breaklines = true,
  columns=[l]{fullflexible},
  xrightmargin = 5mm,
  xleftmargin = 5mm,
  numberstyle = {\ttfamily\scriptsize},
  stepnumber = 1,
  numbersep = 1mm,
  lineskip = -0.5ex,
  showstringspaces = false,
  numbers = left,
  frame = lines,
  backgroundcolor = \color{gray!10},
  rulecolor = \color{black!30},
}
\title{物性物理II}
\date{\today}
\author{\t{harry\_arbrebleu}}
\addbibresource{ref.bib}
\defbibheading{bunken}[\refname]{\section*{#1}}
\begin{document}
\section{1次元の結晶}
$N$個の質量$M$の電子が1次元に並んでいるとする.$s$番目の原子の格子点からのずれを$Q_s$とする.
$s$番目の原子の座標は平衡点での座標を$q^{(0)}$として$q_s = q^{(0)}_s + Q_s$と表される.
平衡点近傍での原子間のポテンシャル$U$をTaylor展開すると,
\begin{align}
  U &= U(q_s^{(0)}, \cdots, q_N^{(0)}) + \frac{1}{2} \sum_{s,s'} \frac{\partial^2}{\partial q_s \partial q_{s'}} U(q_1,\cdots,q_N)\\
  &= U_0 + \frac{1}{2} \sum_{s,s'} U_{ss'} Q_s Q_{s'} + \cdots
\end{align}
である.ここで,原子間隔を$a$とすると$q_s^{(0)} = sa$である.同様に平衡点からの運動量のずれを$P_s$とすると,ハミルトニアンは
\begin{align}
  \hat{H} = \sum_s \frac{\hat{P}_s^2}{2M} + \frac{1}{2} \sum_{s,s'} U_{ss'} \hat{Q}_s \hat{Q}_{s'} + U_0
\end{align}
である.Heisenbergの運動方程式より,
\begin{align}
  \i\hbar  \dot{\hat{Q}}_s &= \qty[\hat{Q}_s,\hat{H}] = \i\hbar\frac{\hat{P}_s}{M}\\
  \i\hbar  \dot{\hat{P}}_s &= \qty[\hat{P}_s,\hat{H}] = -\i\hbar\sum_{s'} U_{ss'} \hat{Q}_{s'}
\end{align}
である.よって,
\begin{align}
  \ddot{\hat{Q}}_s = -\sum_{s'} \frac{U_{ss'}}{M} \hat{Q}_{s'} \label{eom-q}
\end{align}
を得る.また,原子に働く力が原子間距離のみに依存すると仮定すると,$U_{ss'} = U_{s-s'} = U_{s's}$となる.

$s$番目の原子の運動方程式が求められたので,次に,基準振動を求める.$Q_s = u_s e^{-\i\omega t}$とおき,\refe{eom-q}に代入する.
\begin{align}
  -M\omega^2 u_s + \sum_{s'} U(s-s')u_{s'} = 0
\end{align}
を得る.ここで,Blochの定理より$u_s = A\e^{\i ksa}$であるので,これを使うと,
\begin{align}
  \omega^2 &= \frac{1}{M} \sum_{s'} U(s-s') \e^{\i k(s-s')a}\\
  &= \frac{1}{M} \sum_{s'} U(s - s') \cos(k(s-s')a)
\end{align}
と,基準振動の角振動数が求まる.さらに,$U$は隣接する原子間のみに作用することにするt.つまり,
$U$は$s - s' = \pm 1$のときのみゼロでないとする.よって,
\begin{align}
  U(1) = U(-1) = -\frac{1}{2} U(0)
\end{align}
を得る.また,周期的境界条件より,
\begin{align}
  k = \frac{2\pi}{Na}n
\end{align}
である.以上より,
\begin{align}
  \omega^2 = \frac{U(0)}{M} \qty(1 - \cos(ka))
\end{align}
が求まる.

以上を用いて量子化すると,
\begin{align}
  \hat{Q}_s(t) &= \sum_{k} \sqrt{\frac{\hbar}{2NM\omega_k}} \qty(\hat{a}_k \e^{\i ksa} \e^{-\i\omega_k t} + \hat{a}_k^\dagger \e^{-\i ksa} \e^{\i\omega_k t})\\
  \hat{P}_s(t) &= -\sum_{k} \i\sqrt{\frac{\hbar M \omega_k}{2N}} \qty(\hat{a}_k \e^{\i ksa} \e^{-\i\omega_k t} - \hat{a}_k^\dagger \e^{-\i ksa} \e^{\i\omega_k t})
\end{align}
となる.このとき,ハミルトニアンは
\begin{align}
  \hat{H} = \sum_{k} \hbar \omega_k \qty(\hat{a}_k^\dagger \hat{a}_k + \frac{1}{2})
\end{align}
となる.このハミルトニアンの固有エネルギーは
\begin{align}
  \epsilon_k = \hbar\omega_k \qty(n_k + \frac{1}{2})
\end{align}
で,エネルギーが離散化されていることがわかる.
また,上記の生成消滅演算子は
\begin{align}
  \qty[\hat{a}_k,\hat{a}_{k'}^\dagger] = \delta_{kk'}\\
  \qty[\hat{a}_k,\hat{a}_{k'}] = \qty[\hat{a}_k^\dagger,\hat{a}_{k'}^\dagger] = 0
\end{align}
を満たしているのでボゾンであることがわかる.このように,エネルギー$\hbar\omega_k$をもった量子を\textbf{フォノン}と呼ぶ.

\section{3次元の結晶}
上記の議論を3次元に拡張する.3次元結晶中の原子の位置は並進ベクトル$\bm{R}_l = l_1\bm{a}_1 + l_2\bm{a}_2 + l_3\bm{a}_3$で表される.
単位胞中の原子の数を$r$とする.$\kappa$番目の原子の平衡点からの位置のずれを$\bm{\xi}^{\kappa}(\bm{R}_l,t)$とする.
$\bm{\xi}^{\kappa}$の運動方程式は同様に,
\begin{align}
  m_{\kappa}\ddot{\xi}_i^{\kappa}(\bm{R}_l,t) = - \sum_{\bm{R}_{l'}} \sum_{\kappa'} \sum_{j} U_{ij}^{\kappa\kappa'}(\bm{R}_l - \bm{R}_{l'}){\xi_j^{\kappa'}}(\bm{R}_{l'},t) \label{eom-3d}
\end{align}
である.
ここで,Blochの定理より,$\bm{\xi}$は
\begin{align}
  \bm{\xi}^{\kappa}(\bm{R}_l,t) \sim \bm{u}^{\kappa} (\bm{k},J)\e^{\i \bm{k}\cdot\bm{R}_l} \e^{-\i \omega_J t} \label{bloch}
\end{align}
と表される.以下でわかるように,$J$はモードを表し,$J3r$個の値をとる.

\refe{eom-3d}に\refe{bloch}を代入すると,
\begin{align}
  -\omega_J^2 m_\kappa u_{i}^{\kappa} + \sum_{\kappa'} \sum_{j} \qty(\sum_{\bm{R}_{l'}} U_{ij}^{\kappa\kappa'}(\bm{R}_l - \bm{R}_{l'})\e^{-\i\bm{k}\cdot(\bm{R}_l-\bm{R}_{l'})}) u_j^{\kappa'}=0
\end{align}
を得る.上式が恒等的にに0出ない条件は
\begin{align}
  \det(-\omega_J^2 m_{\kappa}\delta_{\kappa\kappa'}\delta_{ij} + \qty(\sum_{\bm{R}_{l'}} U_{ij}^{\kappa\kappa'}(\bm{R}_l - \bm{R}_{l'})\e^{-\i\bm{k}\cdot(\bm{R}_l-\bm{R}_{l'})})) = 0 \label{det}
\end{align}
である.ここで,$i$はベクトルの成分を表し,$i=1,2,3$をとるのであった.また,$\kappa$は単位胞中の原子の数を表し,$r$個の値を取るのであった.
よって,上の行列式は$3r\times 3r$だから,$3r$個の解をもつ.したがって,モードの数は$3r$である.ここで,$3r$個のモードのうち,3個のモードは$\bm{k}\to0$とすると$\omega\to\bm{0}$となる音響モードである.残りの$3(r-1)$は$\bm{k}\to\bm{0}$で$\omega\to0$とならない光学モードである.
これは原子の相対的な運動によるものである.

\refe{det}の行列式を解くと基準振動が求まる.基準振動が求まったものとして量子化を行うと,
\begin{align}
  \hat{\bm{\xi}} (\bm{R}_l,t) = \sqrt{\hbar}\sum_{\bm{k}}\sum_{J} \hat{a}_J (\bm{k})\bm{u}^{\kappa}(\bm{k},J)\e^{\i\bm{k}\cdot\bm{R}_l}\e^{-\i\omega_J(\bm{k})t} + \hat{a}^{\dagger}_{J}(\bm{k})\bm{v}^{\kappa}(\bm{k},J)\e^{-\i\bm{k}\cdot\bm{R}_l}\e^{\i\omega_J(\bm{k})t}
\end{align}
を得る.ここで,$\bm{\xi}$は実であるため,エルミート性を保つために$\bm{v}=\bm{u}^{*}$とする.

\paragraph*{参考文献}
高橋康,物性研究者のための場の量子論1,培風館,1974

\section{フォノンの角運動量}

\section{熱勾配とフォノンの角運動量}

\end{document}