\documentclass{report}
\input{head.tex}
\begin{document}
\section{1次元の結晶}
$N$個の質量$M$の電子が1次元に並んでいるとする.$s$番目の原子の格子点からのずれを$Q_s$とする.
$s$番目の原子の座標は平衡点での座標を$q^{(0)}$として$q_s = q^{(0)}_s + Q_s$と表される.
平衡点近傍での原子間のポテンシャル$U$をTaylor展開すると,
\begin{align}
  U &= U(q_s^{(0)}, \cdots, q_N^{(0)}) + \frac{1}{2} \sum_{s,s'} \frac{\partial^2}{\partial q_s \partial q_{s'}} U(q_1,\cdots,q_N)\\
  &= U_0 + \frac{1}{2} \sum_{s,s'} U_{ss'} Q_s Q_{s'} + \cdots
\end{align}
である.ここで,原子間隔を$a$とすると$q_s^{(0)} = sa$である.同様に平衡点からの運動量のずれを$P_s$とすると,ハミルトニアンは
\begin{align}
  \hat{H} = \sum_s \frac{\hat{P}_s^2}{2M} + \frac{1}{2} \sum_{s,s'} U_{ss'} \hat{Q}_s \hat{Q}_{s'} + U_0
\end{align}
である.Heisenbergの運動方程式より,
\begin{align}
  \i\hbar  \dot{\hat{Q}}_s &= \qty[\hat{Q}_s,\hat{H}] = \i\hbar\frac{\hat{P}_s}{M}\\
  \i\hbar  \dot{\hat{P}}_s &= \qty[\hat{P}_s,\hat{H}] = -\i\hbar\sum_{s'} U_{ss'} \hat{Q}_{s'}
\end{align}
である.よって,
\begin{align}
  \ddot{\hat{Q}}_s = -\sum_{s'} \frac{U_{ss'}}{M} \hat{Q}_{s'} \label{eom-q}
\end{align}
を得る.また,原子に働く力が原子間距離のみに依存すると仮定すると,$U_{ss'} = U_{s-s'} = U_{s's}$となる.

$s$番目の原子の運動方程式が求められたので,次に,基準振動を求める.$Q_s = u_s e^{-\i\omega t}$とおき,\refe{eom-q}に代入する.
\begin{align}
  -M\omega^2 u_s + \sum_{s'} U(s-s')u_{s'} = 0
\end{align}
を得る.ここで,Blochの定理より$u_s = A\e^{\i ksa}$であるので,これを使うと,
\begin{align}
  \omega^2 &= \frac{1}{M} \sum_{s'} U(s-s') \e^{\i k(s-s')a}\\
  &= \frac{1}{M} \sum_{s'} U(s - s') \cos(k(s-s')a)
\end{align}
と,基準振動の角振動数が求まる.さらに,$U$は隣接する原子間のみに作用することにするt.つまり,
$U$は$s - s' = \pm 1$のときのみゼロでないとする.よって,
\begin{align}
  U(1) = U(-1) = -\frac{1}{2} U(0)
\end{align}
を得る.また,周期的境界条件より,
\begin{align}
  k = \frac{2\pi}{Na}n
\end{align}
である.以上より,
\begin{align}
  \omega^2 = \frac{U(0)}{M} \qty(1 - \cos(ka))
\end{align}
が求まる.

以上を用いて量子化すると,
\begin{align}
  \hat{Q}_s(t) &= \sum_{k} \sqrt{\frac{\hbar}{2NM\omega_k}} \qty(\hat{a}_k \e^{\i ksa} \e^{-\i\omega_k t} + \hat{a}_k^\dagger \e^{-\i ksa} \e^{\i\omega_k t})\\
  \hat{P}_s(t) &= -\sum_{k} \i\sqrt{\frac{\hbar M \omega_k}{2N}} \qty(\hat{a}_k \e^{\i ksa} \e^{-\i\omega_k t} - \hat{a}_k^\dagger \e^{-\i ksa} \e^{\i\omega_k t})
\end{align}
となる.このとき,ハミルトニアンは
\begin{align}
  \hat{H} = \sum_{k} \hbar \omega_k \qty(\hat{a}_k^\dagger \hat{a}_k + \frac{1}{2})
\end{align}
となる.このハミルトニアンの固有エネルギーは
\begin{align}
  \epsilon_k = \hbar\omega_k \qty(n_k + \frac{1}{2})
\end{align}
で,エネルギーが離散化されていることがわかる.
また,上記の生成消滅演算子は
\begin{align}
  \qty[\hat{a}_k,\hat{a}_{k'}^\dagger] = \delta_{kk'}\\
  \qty[\hat{a}_k,\hat{a}_{k'}] = \qty[\hat{a}_k^\dagger,\hat{a}_{k'}^\dagger] = 0
\end{align}
を満たしているのでボゾンであることがわかる.このように,エネルギー$\hbar\omega_k$をもった量子を\textbf{フォノン}と呼ぶ.

\section{3次元の結晶}
上記の議論を3次元に拡張する.3次元結晶中の原子の位置は並進ベクトル$\bm{R}_l = l_1\bm{a}_1 + l_2\bm{a}_2 + l_3\bm{a}_3$で表される.
単位胞中の原子の数を$r$とする.$\kappa$番目の原子の平衡点からの位置のずれを$\bm{\xi}^{\kappa}(\bm{R}_l,t)$とする.
$\bm{\xi}^{\kappa}$の運動方程式は同様に,
\begin{align}
  m_{\kappa}\ddot{\xi}_i^{\kappa}(\bm{R}_l,t) = - \sum_{\bm{R}_{l'}} \sum_{\kappa'} \sum_{j} U_{ij}^{\kappa\kappa'}(\bm{R}_l - \bm{R}_{l'}){\xi_j^{\kappa'}}(\bm{R}_{l'},t) \label{eom-3d}
\end{align}
である.
ここで,Blochの定理より,$\bm{\xi}$は
\begin{align}
  \bm{\xi}^{\kappa}(\bm{R}_l,t) \sim \bm{u}^{\kappa} (\bm{k},J)\e^{\i \bm{k}\cdot\bm{R}_l} \e^{-\i \omega_J t} \label{bloch}
\end{align}
と表される.以下でわかるように,$J$はモードを表し,$J3r$個の値をとる.

\refe{eom-3d}に\refe{bloch}を代入すると,
\begin{align}
  -\omega_J^2 m_\kappa u_{i}^{\kappa} + \sum_{\kappa'} \sum_{j} \qty(\sum_{\bm{R}_{l'}} U_{ij}^{\kappa\kappa'}(\bm{R}_l - \bm{R}_{l'})\e^{-\i\bm{k}\cdot(\bm{R}_l-\bm{R}_{l'})}) u_j^{\kappa'}=0
\end{align}
を得る.上式が恒等的にに0出ない条件は
\begin{align}
  \det(-\omega_J^2 m_{\kappa}\delta_{\kappa\kappa'}\delta_{ij} + \qty(\sum_{\bm{R}_{l'}} U_{ij}^{\kappa\kappa'}(\bm{R}_l - \bm{R}_{l'})\e^{-\i\bm{k}\cdot(\bm{R}_l-\bm{R}_{l'})})) = 0 \label{det}
\end{align}
である.ここで,$i$はベクトルの成分を表し,$i=1,2,3$をとるのであった.また,$\kappa$は単位胞中の原子の数を表し,$r$個の値を取るのであった.
よって,上の行列式は$3r\times 3r$だから,$3r$個の解をもつ.したがって,モードの数は$3r$である.ここで,$3r$個のモードのうち,3個のモードは$\bm{k}\to0$とすると$\omega\to\bm{0}$となる音響モードである.残りの$3(r-1)$は$\bm{k}\to\bm{0}$で$\omega\to0$とならない光学モードである.
これは原子の相対的な運動によるものである.

\refe{det}の行列式を解くと基準振動が求まる.基準振動が求まったものとして量子化を行うと,
\begin{align}
  \hat{\bm{\xi}} (\bm{R}_l,t) = \sqrt{\hbar}\sum_{\bm{k}}\sum_{J} \hat{a}_J (\bm{k})\bm{u}^{\kappa}(\bm{k},J)\e^{\i\bm{k}\cdot\bm{R}_l}\e^{-\i\omega_J(\bm{k})t} + \hat{a}^{\dagger}_{J}(\bm{k})\bm{v}^{\kappa}(\bm{k},J)\e^{-\i\bm{k}\cdot\bm{R}_l}\e^{\i\omega_J(\bm{k})t}
\end{align}
を得る.ここで,$\bm{\xi}$は実であるため,エルミート性を保つために$\bm{v}=\bm{u}^{*}$とする.

\paragraph*{参考文献}
高橋康,物性研究者のための場の量子論1,培風館,1974

\section{フォノンの角運動量}

\section{熱勾配とフォノンの角運動量}

\end{document}