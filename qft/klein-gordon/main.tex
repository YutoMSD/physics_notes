\documentclass{report}
\input{../head.tex}
\begin{document}
  \section{Klein-Gordon方程式の「導出」}
    以下では断りの無い限り,$c = \hbar = 1$なる自然単位系を用いる.
    非相対論的・古典的エネルギーの関係式,
    \begin{align}
      E = \frac{\bm{p}^2}{2m}
    \end{align}
    の両辺に波動函数$\psi(t, \bm{x})$掛けて
    \begin{align}
      E &\to \i\pdv{t} \label{energy-to-diff}\\ 
      \bm{p} &\to -\i\grad \label{momentum-to-diff}
    \end{align}
    なる変換を行えば,
    \begin{align}
      E &= \frac{\bm{p}^2}{2m} \\ 
      \Rightarrow E\psi(t, \bm{x}) &= \frac{\bm{p}^2}{2m}\psi(t, \bm{x}) \\ 
      \Rightarrow \i\pdv{t}\psi(t, \bm{x}) &= \frac{1}{2m}\qty(-\i\grad)^2\psi(t, \bm{x}) \\ 
      \Rightarrow \i\pdv{t}\psi(t, \bm{x}) &= -\frac{1}{2m}\laplacian\psi
    \end{align}
    となり,Schr\"odinger方程式を得る.
    \par
    では相対論的なエネルギーの関係式,
    \begin{align}
      E^2 = \bm{p}^2 + m^2\label{relativity-energy}
    \end{align}
    を変換すると,どのようになるだろう.
    自然単位系を用いているため,静止エネルギーの2乗について$m^2c^4 = m^2$となっていることに注意する.
    \refe{relativity-energy}の両辺に波動函数$\psi(t, \bm{x})$掛けて,
    \refe{energy-to-diff},\refe{momentum-to-diff}を用いれば,
    \begin{align}
      E^2 &= \bm{p}^2 + m^2 \\ 
      \Rightarrow E\psi(t, \bm{x}) &= \frac{\bm{p}^2}{2m}\psi(t, \bm{x}) + m^2\psi(t, \bm{x}) \\ 
      \qty(\i\pdv{t})^2\psi(t, \bm{x}) &= \qty[\qty(-\i\grad)^2 + m^2]\psi(t, \bm{x}) \\ 
      -\pdv[2]{t}\psi(t, \bm{x}) &= -\laplacian\psi(t, \bm{x}) + m^2\psi(t, \bm{x}) \\ 
      \qty(\partial_{\mu}\partial^{\mu} - m^2)\psi(t, \bm{x}) &= 0
    \end{align}
    が成立する.
\end{document}