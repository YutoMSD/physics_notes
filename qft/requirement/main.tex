\documentclass{report}
\input{../head.tex}
\begin{document}
  本章では,これから議論する場の量子論の準備を行う.
  場の量子論は,既存の量子力学などの物理法則あるいは方程式を\textbf{Poincar\'e変換}に対して不変な形に書き直す理論である.
  ただし,Poincar\'e変換は,Lorentz変換と時空並進変換のことである.
  \section{記法}
    まず,微小な時空間上の2点,$\qty(ct, x, y, z)$,$\qty(ct + \dd{t}, x + \dd{x}, y + \dd{y}, z + \dd{z})$に対して,世界長さ$\dd{s^2}$を考える.
    \begin{align}
      \dd{s^2} \coloneqq c^2\dd{t}^2 - \qty(\dd{x}^2 + \dd{y}^2 + \dd{z}^2)
    \end{align}
    4次元時空座標を,
    \begin{align}
      x^{\mu} &\coloneqq \qty(ct, x, y, z) \\ 
      x_{\nu} &\coloneqq \qty(ct, -x, -y, -z)
    \end{align}
    と定義する.
    Einsteinの縮約を使っていることに注意する.
    計量テンソル$\eta_{\mu\nu}$,$\eta^{\mu\nu}$を,
    \begin{align}
      x_{\nu} &= \eta_{\mu\nu}x^{\mu} \\ 
      x^{\nu} &= x_{\mu}\eta^{\mu\nu}
    \end{align}
    となるように定義する.
    計量テンソルを用いれば,
    \begin{align}
      \dd{s^2} &= \eta_{\mu\nu}\dd{x^{\mu}}\dd{x^{\nu}} \label{def-of-proper-length} \\ 
      &= \dd{x_{\mu}}\dd{x_{\nu}}\eta^{\mu\nu}
    \end{align}
    と書ける.
    $\eta_{\mu\nu}$は,上付き添え字が$k$個,下付き添え字が$l$個あるものに対して,添え字を上付き添え字を$k + l - 2$個,下付き添え字を$k + l + 2$個にするものだと考えてよい.
    なお,負の添え字の数は,添え字の上下を逆転させたものと考える.
    同様に,$\eta^{\mu\nu}$は,上付き添え字が$k$個,下付き添え字が$l$個あるものに対して,添え字を上付き添え字を$k + l + 2$個,下付き添え字を$k + l - 2$個にするものだと考えてよい.
  \section{Poincar\'e変換}
    Poincar\'e変換は,Lorentz変換のパラメータを$\Lambda_{\nu}^{\mu}$,時空並進のパラメータを$a^{\mu}$とすると,
    \begin{align}
      x'^{\mu} = \Lambda_{\nu}^{\mu}x^{\nu} + a^{\mu}
    \end{align}
    と書ける.
    微小変位は,
    \begin{align}
      \dd{x'^{\mu}} &= \dd{\qty(\Lambda_{\nu}^{\mu}x^{\nu} + a^{\mu})} \\ 
      &= \Lambda_{\nu}^{\mu}\dd{x^{\nu}}
    \end{align}
    と書けるから,世界長さ$\dd{s^2}$は,
    \begin{align}
      \dd{s'^2} &= \eta_{\rho\lambda}\dd{x'^{\rho}}\dd{x'^{\lambda}} \\ 
      &= \eta_{\rho\lambda}\qty(\Lambda_{\nu}^{\rho}\dd{x^{\nu}})\qty(\Lambda_{\mu}^{\lambda}\dd{x^{\mu}}) \\ 
      &= \eta_{\rho\lambda}\Lambda_{\nu}^{\rho}\Lambda_{\mu}^{\lambda}\dd{x^{\nu}}\dd{x^{\mu}}
    \end{align}
    となる.
    今,Poincar\'e変換に対して方程式は不変であることが要請されているのであった.
    \refe{def-of-proper-length}で与えられる世界長さを与える方程式もPoincar\'e変換に対して不変であるべきだから,
    \begin{align}
      \dd{s'^2} &= \dd{s^2} \\ 
      \Leftrightarrow \eta_{\rho\lambda}\Lambda_{\nu}^{\rho}\Lambda_{\mu}^{\lambda}\dd{x^{\nu}}\dd{x^{\mu}} &= \eta_{\mu\nu}\dd{x^{\mu}}\dd{x^{\nu}} \\ 
      \Leftrightarrow \eta_{\rho\lambda}\Lambda_{\nu}^{\rho}\Lambda_{\mu}^{\lambda} &= \eta_{\mu\nu} \\ 
      \Leftrightarrow \eta_{\rho\lambda} &= \eta_{\mu\nu}\qty(\Lambda^{-1})_{\rho}^{\nu}\qty(\Lambda^{-1})_{\lambda}^{\mu}
    \end{align}
    である.
    ただし,
    \begin{align}
      \Lambda_{\nu}^{\rho}\qty(\Lambda^{-1})_{\rho}^{\nu} = \qty(\Lambda^{-1})_{\rho}^{\nu}\Lambda_{\nu}^{\rho} = 1
    \end{align}
    なる関係を用いた.
  \section{スカラー・ベクトル・テンソル}
    本節ではスカラー・ベクトル・テンソルを定義する.
    Lorentz変換のパラメータを$\Lambda$とする.
    \subsection{スカラー}
      スカラーはLorentz変換に対して不変な量である.
      すなわち,
      \begin{align}
        S \mapsto S \eqqcolon S'
      \end{align}
      なる量である.
    \subsection{ベクトル}
      ベクトルは2種類あり,Lorentz変換によって時空座標を変換したときに,時空座標$x^{\mu}$と同じように変換される反変ベクトルと,
      $x_{\mu}$と同じように変換される共変ベクトルに分けられる.
      すなわち,
      \begin{align}
        A^{\mu} &\mapsto \Lambda_{\nu}^{\mu}A^{\nu} \eqqcolon A'^{\mu} \\ 
        B_{\mu} &\mapsto B_{\nu}\qty(\Lambda^{-1})_{\mu}^{\nu} \eqqcolon B'_{\mu}
      \end{align}
      において,$A^{\mu}$,$A'^{\mu}$が反変ベクトル,$B_{\mu}$,$B'_{\mu}$が共変ベクトルである.
    \subsection{テンソル}
      テンソルは3種類あり,Lorentz変換によって時空座標を変換したときに,時空座標$x^{\mu}$を2回変換したとき同じように変換される2階の反変テンソル,
      時空座標$x^{\mu}$を1回変換してから1回逆変換したとき同じように変換される2階の混合テンソル,
      時空座標$x_{\mu}$を2回変換したとき同じように変換される2階の共変テンソルの3つに分けられる.
      すなわち,
      \begin{align}
        T^{\mu\nu} &\mapsto \Lambda_{\rho}^{\mu}\Lambda_{\lambda}^{\nu}T^{\rho\lambda} \eqqcolon T'^{\mu\nu} \\ 
        T_{\nu}^{\mu} &\mapsto \Lambda_{\rho}^{\mu}T_{\lambda}^{\rho}\qty(\Lambda^{-1})_{\nu}^{\lambda} \eqqcolon {T'}_{\nu}^{\mu} \\ 
        T_{\nu\mu} &\mapsto T_{\lambda}^{\rho}\qty(\Lambda^{-1})_{\mu}^{\rho}\qty(\Lambda^{-1})_{\nu}^{\lambda} \eqqcolon T'_{\nu\mu}
      \end{align}
      の3つがある.
      3つのテンソルの間には,
      \begin{align}
        T^{\mu\nu} = T_{\lambda}^{\mu}\eta^{\nu\lambda} = T_{\rho\lambda}\eta^{\mu\rho}\eta^{\nu\lambda}
      \end{align}
      なる関係がある.
\end{document}