\documentclass{report}
\input{../head.tex}
\begin{document}
  本章では,これから議論する場の量子論の準備を行う.
  場の量子論は,既存の量子力学などの物理法則あるいは方程式を\textbf{Poincar\'e変換}に対して不変な形に書き直す理論である.
  ただし,Poincar\'e変換は,Lorentz変換と時空並進変換のことである.
  \par
  まず,この要請から微小な時空間上の2点,$\qty(ct, x, y, z)$,$\qty(ct + \dd{t}, x + \dd{x}, y + \dd{y}, z + \dd{z})$に対して,世界長さ$\dd{s^2}$を考える.
  \begin{align}
    \dd{s^2} \coloneqq c\dd{t}^2 - \qty(\dd{x}^2 + \dd{y}^2 + \dd{z}^2)
  \end{align}
  4次元時空座標を,
  \begin{align}
    x^{\mu} &\coloneqq \qty(ct, x, y, z) \\ 
    x_{\nu} &\coloneqq \qty(ct, -x, -y, -z)
  \end{align}
  と定義する.
  Einsteinの縮約を使っていることに注意する.
  計量テンソル$\eta_{\mu\nu}$を,
  \begin{align}
    \eta_{\mu\nu} \coloneqq \mqty(
      1 & 0 & 0 & 0 \\ 
      0 & -1 & 0 & 0 \\ 
      0 & 0 & -1 & 0 \\ 
      0 & 0 & 0 & -1
    )
  \end{align}
  と定義する.
  これらの道具を用いれば,
  \begin{align}
    \dd{s^2} = \eta_{\mu\nu}\dd{x^{\mu}}\dd{x^{\nu}}
  \end{align}
  と書ける.
  さて,世界長さ$\dd{s}$がPoincar\'e変換に対して不変であることを示そう.
\end{document}