\documentclass{report}
\input{../head.tex}
\begin{document}
1粒子の量子力学では波動関数を導入し,位置演算子と運動量演算子の非可換性から量子化を行った.
場の量子論では粒子数が量子的に変化しうるため,波動関数を生成消滅演算子で表す必要がある.
生成消滅演算子の非可換性を用いた2度目の量子化を\textbf{第2量子化}(second quantization)という.

時空点$(\bm{r},t)$に粒子を1つ生成する演算子を$\hat{\psi}^{\dagger}(\bm{r},t)$と表し,
消滅させる演算子を$\hat{\psi}(\bm{r},t)$と表す.それぞれ粒子の\textbf{生成演算子},\textbf{消滅演算子}という.これらは
\textbf{粒子数演算子}$\hat{n}(\bm{r},t)$とは非可換である.なぜなら,粒子を生成してから粒子数を測定する操作$\hat{n}\hat{\psi}^{\dagger}$と
粒子数を測定してから粒子を生成する操作$\hat{\psi}^{\dagger}\hat{n}$では,結果として得られる粒子数は1異なるからである.これを数式で表すと,
同じ時空点の演算子の間には
\begin{align}
  \hat{n}\hat{\psi}^{\dagger} - \hat{\psi}^{\dagger}\hat{n} = \hat{\psi}^{\dagger}
\end{align}
が要請される.同様に,
\begin{align}
  \hat{n}\hat{\psi} - \hat{\psi}\hat{n} = -\hat{\psi}
\end{align}
が要請される.よって,これらを満たす$\hat{n}$と$\hat{\psi},\hat{\psi}^{\dagger}$の間には
\begin{align}
  &\hat{n} = \hat{\psi}^{\dagger}\hat{\psi}\\
  &\begin{dcases}
    \qty[\hat{\psi}^{\dagger},\hat{\psi}] = -1\\
    \qty[\hat{\psi},\hat{\psi}] = \qty[\hat{\psi}^{\dagger},\hat{\psi}^{\dagger}] = 0
  \label{boson-com}
  \end{dcases}
\end{align}
または
\begin{align}
  &\hat{n} = \hat{\psi}^{\dagger}\hat{\psi}\\
  &\begin{dcases}
    \qty{\hat{\psi}^{\dagger},\hat{\psi}} = 1\\
    \qty{\hat{\psi},\hat{\psi}} = \qty{\hat{\psi}^{\dagger},\hat{\psi}^{\dagger}} = 0
  \label{fermion-com}
  \end{dcases}
\end{align}
が満たされる必要がある.後で見るように\refe{boson-com}はボゾンを表し,\refe{fermion-com}はフェルミオンを表す.

まず,\refe{boson-com}を満たす演算子を考える.\refe{boson-com}は時空の場所依存性を書くと
\begin{align}
  \qty[\hat{\psi}^\dagger(\bm{r}',t'),\hat{\psi}(\bm{r},t)] &= -\delta(\bm{r} - \bm{r}')\delta(t - t')\\
  \qty[\hat{\psi}(\bm{r}',t'),\hat{\psi}(\bm{r},t)] &= \qty[\hat{\psi}^{\dagger}(\bm{r}',t'),\hat{\psi}^{\dagger}(\bm{r},t)] = 0
\end{align}
となる.

1粒子系のハミルトニアン
\begin{align}
  \hat{H}_1 = -\frac{\hbar^2}{2m}\nabla^2 + V(\bm{r})
\end{align}
を多粒子系に拡張する.多粒子のハミルトニアンとして
\begin{align}
  \hat{H} = \int \dd{\bm{r}} \hat{\psi}^{\dagger}(\bm{r}) \qty(-\frac{\hbar^2}{2m}\nabla^2 + V(\bm{r}))\hat{\psi}(\bm{r})
\end{align}
を考える.上式は部分積分を実行することで
\begin{align}
  \hat{H} = \int \dd{\bm{r}} \qty(\frac{\hbar^2}{2m}(\nabla\hat{\psi}^{\dagger})(\nabla\hat{\psi}) + V(\bm{r})\hat{\psi}^{\dagger}\hat{\psi}) \label{hamiltonian}
\end{align}
と変形できる.

上のハミルトニアンが1粒子の量子力学を再現することを確認する.まず,真空を
\begin{align}
  \hat{\psi}(\bm{r})\ket{0} = 0
\end{align}
として定義する.つまり,粒子が存在しないという状態である.また,粒子が$\bm{r}$に存在する状態$\ket{\bm{r}}$は
\begin{align}
  \ket{\bm{r}} = \hat{\psi}^{\dagger}(\bm{r}) \ket{0}
\end{align}
と表される.点$\bm{r}$に粒子が存在する確率振幅を$\phi(\bm{r},t)$とすると1粒子状態は
\begin{align}
  \ket{\phi} = \int \dd{\bm{r}} \phi(\bm{r},t) \ket{\bm{r}} = \int \dd{\bm{r}} \phi(\bm{r},t) \hat{\psi}^{\dagger}(\bm{r})\ket{0} \label{1-state}
\end{align}
と表される.\refe{1-state}に\refe{hamiltonian}を作用させると,
\begin{align}
  \hat{H}\ket{\phi} = \int \dd{\bm{r}} \int \dd{\bm{r}'} \phi(\bm{r},t) \qty(\frac{\hbar^2}{2m}\qty(\nabla_{\bm{r}'}\hat{\psi}^{\dagger}(\bm{r}'))\qty(\nabla_{\bm{r}'}\hat{\psi}(\bm{r}')) + V(\bm{r}')\hat{\psi}^{\dagger}(\bm{r}')\hat{\psi}(\bm{r}'))\hat{\psi}^{\dagger}(\bm{r})\ket{0}
\end{align}
を得る.
さらに第1項を部分積分して
\begin{align}
  \hat{H}\ket{\phi} = \int \dd{\bm{r}} \int \dd{\bm{r}'} \phi(\bm{r},t) \qty(-\frac{\hbar^2}{2m}\qty(\nabla_{\bm{r}'}^2\hat{\psi}^{\dagger}(\bm{r}'))\hat{\psi}(\bm{r}') + V(\bm{r}')\hat{\psi}^{\dagger}(\bm{r}')\hat{\psi}(\bm{r}'))\hat{\psi}^{\dagger}(\bm{r})\ket{0}
\end{align}
を得る.ここで,
\begin{align}
  \hat{\psi}(\bm{r}')\hat{\psi}^{\dagger}(\bm{r}) \ket{0} = \delta(\bm{r}' - \bm{r})
\end{align}
だから,さらに変形することができ,
\begin{align}
  \hat{H}\ket{\phi} &= \int \dd{\bm{r}} \int \dd{\bm{r}'} \phi(\bm{r},t) \qty(-\frac{\hbar^2}{2m}\qty(\nabla_{\bm{r}'}^2\hat{\psi}^{\dagger}(\bm{r}')) + V(\bm{r}')\hat{\psi}^{\dagger}(\bm{r}))\delta(\bm{r}' - \bm{r})\\
  &= \int \dd{\bm{r}} \qty[\qty(-\frac{\hbar^2}{2m}\nabla^2 + V(\bm{r}))\phi(\bm{r},t)]\hat{\psi}^{\dagger}(\bm{r})\ket{0} \label{integral-of-state}
\end{align}
を得る.\refe{integral-of-state}に左から$\bra{\bm{r}}$を作用すると,
\begin{align}
  \bra{\bm{r}}\hat{H}\ket{\phi} = \qty(-\frac{\hbar^2}{2m}\nabla^2 + V(\bm{r}))\phi(\bm{r},t)
\end{align}
となる.上式の左辺はSchrödinger方程式
\begin{align}
  \qty(-\frac{\hbar^2}{2m}\nabla^2 + V(\bm{r}))\phi(\bm{r},t) = \i\hbar\frac{\partial}{\partial t}\phi(\bm{r},t)
\end{align}
の左辺と一致する.よって,
\begin{align}
  \hat{H} \ket{\phi} = \i\hbar\frac{\partial}{\partial t}\ket{\phi}
\end{align}
を得る.これが,多体の量子論の時間発展を記述する方程式である.

次に2粒子系を考える.粒子が$\bm{r}_1,\bm{r}_2$にいる状態は
\begin{align}
  \ket{\bm{r}_1,\bm{r}_2} = \hat{\psi}(\bm{r}_1)^{\dagger}\hat{\psi}(\bm{r}_2)^{\dagger}\ket{0}
\end{align}
である.さらに2粒子状態$\ket{\phi_2\phi_1}$は
\begin{align}
  \ket{\phi_2\phi_1} = \frac{1}{\sqrt{2}}\int\dd{\bm{r}}\int\dd{\bm{r}'} \phi_2(\bm{r}')\phi_1(\bm{r}) \hat{\psi}^{\dagger}(\bm{r}')\hat{\psi}^{\dagger}(\bm{r}) \ket{0}
\end{align}
と表される.よって,$\bm{r}_1,\bm{r}_2$で粒子を観測する確率振幅は
\begin{align}
  \bra{\bm{r}_1\bm{r}_2}\ket{\phi_2\phi_1} &= \frac{1}{\sqrt{2}}\int\dd{\bm{r}}\int\dd{\bm{r}'}\phi_2(\bm{r}')\phi_1(\bm{r})\bra{0}\hat{\psi}(\bm{r}_1)\hat{\psi}(\bm{r}_2)\hat{\psi}^{\dagger}(\bm{r}')\hat{\psi}^{\dagger}(\bm{r}) \ket{0}\\
  &= \frac{1}{\sqrt{2}}\int\dd{\bm{r}}\int\dd{\bm{r}'}\phi_2(\bm{r}')\phi_1(\bm{r})\bra{0}\qty(\delta(\bm{r}_2 - \bm{r}')\hat{\psi}^{\dagger}(\bm{r}) + \delta(\bm{r}_2 - \bm{r})\hat{\psi}^{\dagger}(\bm{r}')) \ket{0}\\
  &= \frac{1}{\sqrt{2}}\int\dd{\bm{r}}\qty(\phi_2(\bm{r}_2)\phi_1(\bm{r}) + \phi_2(\bm{r})\phi_1(\bm{r}_2))\bra{0}\hat{\psi}(\bm{r}_1)\hat{\psi}^{\dagger}(\bm{r})\ket{0}\\
  &= \frac{1}{\sqrt{2}}\qty(\phi_2(\bm{r}_2)\phi_1(\bm{r}_1) + \phi_2(\bm{r}_1)\phi_1(\bm{r}_2))
\end{align}
となる.これは2つの粒子の分布を対称化した確率振幅となっている.つまり,どちらの粒子がどちらの波動関数の状態にあるのか区別できない.
また,これはBose統計に従う粒子,ボゾンを表している.思い出してほしいのは,出発点は\refe{boson-com}であったことである.
\end{document}