\documentclass{report}
\input{../head.tex}
\begin{document}
演算子がどのように第2量子化されるかを見る.

まずは1粒子に作用する演算子を足し合わせた
\begin{align}
  \hat{F} = \sum_i \hat{f}(\bm{r}_i)
\end{align}
が場の演算子を用いてどのように表されるのかを考える.$\hat{f}$の例は,運動エネルギー$\frac{\hat{p}_i^2}{2m}$や中心力ポテンシャル$V(\bm{r}_i)$などである.
位置$\bm{r}_i$固有状態$\lambda$にある粒子の波動関数を$\v_{\lambda}(\bm{r}_i)$と書くことにする.
これを使って多体系全体の状態$\ket{n}$を
\begin{align}
  \ket{n} = \ket{v_\alpha(\bm{r}_1)}\otimes\ket{v_\beta(\bm{r}_2)}\otimes\cdots\otimes\ket{v_\mu(\bm{r}_n)} = \ket{v_\alpha(\bm{r}_1),v_\beta(\bm{r}_2),\cdots,v_\nu(\bm{r}_n)}
\end{align}
と書く.まず,
\begin{align}
  \hat{f}(\bm{r}_i)\ket{v_\lambda(\bm{r}_i)} &= \sum_{\mu}\ket{v_\mu(\bm{r}_i)}\bra{v_\mu(\bm{r}_i)}\hat{f}(\bm{r}_i)\ket{v_\lambda(\bm{r}_i)}\\
  &= \sum_{\mu} \bra{v_\mu(\bm{r}_i)}\hat{f}(\bm{r}_i)\ket{v_\lambda(\bm{r}_i)}\ket{v_\mu(\bm{r}_i)}
\end{align}
が成り立つ.ここで,
\begin{align}
  \bra{v_\mu(\bm{r})}\hat{f}(\bm{r})\ket{v_\lambda(\bm{r})} = \int \dd[3]{\bm{r}}\v_{\mu}^*(\bm{r})\hat{f}(\bm{r})\v_{\lambda}(\bm{r})
\end{align}
である.
これを用いて
\begin{align}
  \hat{f}(\bm{r}_i)\ket{n} &= \hat{f}(\bm{r}_i)\ket{v_\alpha(\bm{r}_1),v_\beta(\bm{r}_2),\cdots,v_\lambda(\bm{r}_i),\cdots,v_\nu(\bm{r}_n)}\\
  &= \ket{v_\alpha(\bm{r}_1)}\otimes\ket{v_\beta(\bm{r}_2)}\otimes\cdots\qty(\hat{f}(\bm{r}_i)\ket{v_\lambda(\bm{r}_i)})\cdots\otimes\ket{v_\nu(\bm{r}_n)}\\
  &= \sum_{\mu} \bra{v_\mu(\bm{r}_i)}\hat{f}(\bm{r}_i)\ket{v_\lambda(\bm{r}_i)}\ket{v_\alpha(\bm{r}_1),v_\beta(\bm{r}_2),\cdots,v_\mu(\bm{r}_i),\cdots,v_\nu(\bm{r}_n)}\\
\end{align}
を得る.ここで,
\begin{align}
  \ket{v_\alpha(\bm{r}_1),v_\beta(\bm{r}_2),\cdots,v_\mu(\bm{r}_i),\cdots,v_\nu(\bm{r}_n)} = \hat{a}_\mu^{\dagger}\hat{a}_\lambda\ket{v_\alpha(\bm{r}_1),v_\beta(\bm{r}_2),\cdots,v_\lambda(\bm{r}_i),\cdots,v_\nu(\bm{r}_n)} =  \hat{a}_\mu^{\dagger}\hat{a}_\lambda\ket{n}
\end{align}
であるので,
\begin{align}
  \hat{F}\ket{n} &= \sum_i \hat{f}(\bm{r}_i)\ket{n}\\
  &= \sum_{i}\sum_{\mu} \bra{v_\mu(\bm{r}_i)}\hat{f}(\bm{r}_i)\ket{v_\lambda(\bm{r}_i)}\hat{a}_\mu^{\dagger}\hat{a}_\lambda\ket{n}\\
  &= \sum_{\mu,\lambda}\sum_{i} \bra{v_\mu(\bm{r})}\hat{f}(\bm{r})\ket{v_\lambda(\bm{r})}\hat{a}_\mu^{\dagger}\hat{a}_\lambda\ket{n}\\
\end{align}
となる.よって,1粒子に作用する演算子は第2量子化により
\begin{align}
  \hat{F} = \sum_{\lambda,\mu} \bra{v_\mu(\bm{r})}\hat{f}(\bm{r})\ket{v_\lambda(\bm{r})}\hat{a}_\mu^{\dagger}\hat{a}_\lambda \label{f}
\end{align}
と書ける.

2つの粒子の相互作用を表す演算子
\begin{align}
  \hat{G} = \sum_{i\neq j} \hat{g}(\bm{r}_i,\bm{r}_j)
\end{align}
の第2量子化も同様に
\begin{align}
  \hat{G} = \frac{1}{2} \sum_{\alpha,\beta,\lambda,\mu} \bra{v_\alpha(\bm{r}_1)v_\beta(\bm{r}_2)}\hat{g}(\bm{r}_1,\bm{r}_2)\ket{v_\lambda(\bm{r}_1)v_\mu(\bm{r}_2)}\hat{a}_\alpha^{\dagger}\hat{a}_\beta^{\dagger}\hat{a}_\mu\hat{a}_\lambda \label{g}
\end{align}
である.

さらに,場の演算子は座標の依存性のみを考えると,
\begin{align}
  \psi(\bm{r}) = \sum_i v_i(\bm{r})\hat{a}_i,\ \psi(\bm{r})^{\dagger} = \sum_i v_i^*(\bm{r})\hat{a}_i^{\dagger} \label{field-v}
\end{align}
と書けるのであった.これを用いると$\hat{F}$と$\hat{G}$は
\begin{align}
  \hat{F} &= \int \dd[3]{\bm{r}}\psi^{\dagger}(\bm{r})\hat{f}(\bm{r})\psi(\bm{r}) \label{f-field} \\
  \hat{G} &= \frac{1}{2}\int\dd[3]{\bm{r}_1}\int\dd[3]{\bm{r}_2}\psi^{\dagger}(\bm{r}_1)\psi^{\dagger}(\bm{r}_2)\hat{g}(\bm{r}_1,\bm{r}_2)\psi(\bm{r}_2)\psi(\bm{r}_1) \label{g-field}
\end{align}
と表される.確かに\refe{f-field}と\refe{g-field}に\refe{field-v}を代入すると,\refe{f}と\refe{g}に一致する.

\end{document}