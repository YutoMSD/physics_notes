\documentclass{report}
\input{../head.tex}
\begin{document}
\section{イオン化エネルギー}
完全な復習なので式だけ示す.古典的な円運動は
\begin{align}
  \frac{m_ev^2}{r} = \frac{1}{4\pi\epsilon_0}\frac{e^2}{r^2}
\end{align}
量子化条件は
\begin{align}
  m_e vr = n\frac{h}{2\pi}
\end{align}
許される軌道半径は
\begin{align}
  r_n = n^2 \frac{h^2\epsilon_0}{\pi m_e e^2}
\end{align}
$n=1$のときの軌道半径をBohr半径と言い,$5.3\times10^{-11}$ mである.

各軌道に対する電子の全エネルギーは
\begin{align}
  E_n = -\frac{m_ee^4}{8\pi\epsilon_0^2h^2}\frac{1}{n^2}
\end{align}
$n=1$のとき,-13.6 eVである.

イオン化エネルギーの定義は
\begin{align}
  \Delta E = E_{\infty} - E_1 = 13.6\ \r{eV}
\end{align}
よって,水素原子のイオン化エネルギーは13.6 eVである.

常温・常圧では2原子あるいはそれ以上の原子が結合して分子の状態にあることが多い.この場合には,解離課程を経て原子になり,その後,電離することが多い.

\section{解離過程}
では,どのようにして電離に必要なエネルギーを与えるのだろうか.主に,電子に依る衝突電離と光電離がある.

\subsection{衝突電離}
イオン化エネルギー以上のエネルギーを衝突によって与えて中性粒子を電離させる方法が衝突電離である.1個の原子を電離するには5-25 eVのエネルギーが必要である.

衝突させる粒子としては荷電粒子が用いられる.なぜなら,電圧により容易に加速させられるからである.電子を使ったものを\textbf{電子衝突電離}(electron impact ionization)という.
電子はイオンに比べ,種になる電子が作りやすく,衝突に依るエネルギーの授受の効率が良いという特徴がある.


\subsection{光電離}

\end{document}