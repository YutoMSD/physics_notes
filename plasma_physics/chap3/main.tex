\documentclass{report}
\input{../head.tex}
\begin{document}
  この章ではプラズマがどのような運動をするかを学ぶ.

  \section{はじめに}

    プラズマ輸送の解析には\textbf{流体モデル}(Fluid Model)\index{りゅうたいもでる@流体モデル}と\textbf{運動論的モデル}(Kinetic Model)\index{うんどうろんてきもでる@運動論的モデル}を用いる.
    前者はプラズマを連続体として扱い,巨視的なプラズマの振る舞いを記述する.後者は粒子個々の運動を解き,微視的なプラズマの振る舞いを記述する.
    
  \section{一様磁場中での粒子の運動}
    \subsection{磁場に垂直方向の運動}
      
      一様磁場中では電荷はサイクロトロン運動\index{さいくろとろんうんどう@サイクロトロン運動}をする.美束密度$\bm{B}$中を電荷$q$を持つ粒子が速度$\bm{v}$で運動しているとする.
      この粒子が受ける力は
      \begin{align}
        \bm{F} = q\bm{v}\times\bm{B}
      \end{align}
      である.ここで,
      \begin{align}
        \bm{F} \cdot \bm{v} = 0
      \end{align}
      なのでLorentz力は仕事をしないことがわかる.

      この粒子の運動方程式は以下のようになる.
      \begin{align}
        m \dv{\bm{v}}{t} = q\bm{v}\times\bm{B}
      \end{align}
      ここで,$\bm{B} = (0,0,B)$,$\bm{v} = (v_x,v_y,v_z)$とする.運動方程式の各成分は
      \begin{align}
        m\dv{v_x}{t} &= qv_yB\\
        m\dv{v_y}{t} &= -qv_xB\\
        m\dv{v_z}{t} &= 0
      \end{align}
      となる.これらを整理すると,
      \begin{align}
        &\ddot{v_x} = \frac{qB}{m}v_x \to \ddot{v_x} = - \omega_{\r{c}}^2 v_x\\
        &\ddot{v_y} = -\frac{qB}{m}v_y \to \ddot{v_y} = \omega_{\r{c}}^2 v_y
      \end{align}
      となる.よって,荷電粒子は$xy$平面上を\textbf{サイクロトロン周波数}(cyclotron frequency)$\omega_{\r{c}}=qB/m$\index{さいくろとろんしゅうはすう@サイクロトロン周波数}で円運動することがわかる\footnote{
        通常,サイクロトロン周波数という場合には,電荷$q$の絶対値をとって,$\omega_{\r{c}} = |q|B/m$とすることが多い.
      }
      .また,初期条件
      \begin{align}
        \begin{dcases}
          x = 0\\
          y = 0\\
          v_x = v_{\perp}\\
          v_y = 0
        \end{dcases}
        \ \r{at}\ t = 0
      \end{align}
      の下で,この粒子の軌道は
      \begin{align}
        \begin{dcases}
          x(t) = \frac{v_{\perp}}{\omega_{\r{c}}}\sin(\omega_{\r{c}}t)\\
          y(t) = \frac{v_{\perp}}{\omega_{\r{c}}}(\cos(\omega_{\r{c}}t)-1)
        \end{dcases}
      \end{align}
      となる.したがって,
      \begin{align}
        x^2 + \qty(y + \frac{v_{\perp}}{\omega_{\r{c}}})^2 = \qty(\frac{v_{\perp}}{\omega_{\r{c}}})^2
      \end{align}
      が成り立つので,この粒子の円運動の\textbf{旋回中心}\index{せんかいちゅうしん@旋回中心}は
      \begin{align}
        (x_g,y_g) = (0, -\frac{v_{\perp}}{\omega_{\r{c}}})
      \end{align}
      で\textbf{旋回半径}\index{せんかいはんけい@旋回半径}は
      \begin{align}
        r_{\r{L}} = \frac{v_{\perp}}{\abs{\omega_{\r{c}}}}
      \end{align}
      である.

      まとめると,電荷数$Z$の粒子のサイクロトロン周波数は
      \begin{align}
        \omega_{\r{c}j} = \frac{Z e B}{m_j}
      \end{align}
      である\footnote{$j$はイオンか電子かを示す.}.サイクロトロン半径は
      \begin{align}
        r_{\r{L}j} = \frac{m_j v_{\perp}}{ZeB}
      \end{align}
      である.また,熱速度$v_{{\r{th}}j}$を用いて
      \begin{align}
        r_{\r{L}j} = \frac{m_j v_{{\r{th}}j}}{ZeB}
      \end{align}
      と表すことが多い.

      $z$軸方向には力は働かないので
      \begin{align}
        v_{\parallel} &= \dot{v_z} = \r{const.}\\
        z &= z_0 + v_{\parallel}t
      \end{align}
      である.つまり,全体としてはらせん運動をしている.

    \subsection{ドリフト運動}
      
      ここでは一様磁場に加えて電場や重力などが働く場合の粒子の運動や磁場が空間的に変化する場合の粒子の運動を考える.

      \subsubsection{$E\times B$ドリフト}
        磁場が$z$軸方向に向いているとする.前節とり粒子は$xy$平面上を円運動するのであった.ここに電場を$x$方向に加える.
        このとき,粒子は一方では加速されもう一方では減速される.つまり,上方では$V_0+V_1$,下方では$V_0-V_1$のように異なってしまう.
        速度が$V_0+V_1$のときは旋回半径が大きく,$V_0-V_1$のときは旋回半径が小さくなる.よって,旋回中心が
        下方へドリフトしていく.

        このときのドリフト速度は
        \begin{align}
          \bm{v}_{\r{g}E} = \frac{\bm{E}\times\bm{B}}{B^2} 
        \end{align}
        である.以下に導出を行う.
        
        まず,粒子の運動方程式は
        \begin{align}
          m\dv{\bm{v}}{t} = q\bm{v}\times\bm{B} + q\bm{E} \label{eom-cyclotron}
        \end{align}
        である.この運動を旋回中心の運動と旋回中心のまわりのLarmor運動に分ける.
        \begin{align}
          \bm{r} &= \bm{r}_{\r{gc}} + \bm{r}_{\r{c}}\\
          \bm{v} &= \bm{v}_{\r{gc}} + \bm{v}_{\r{c}} \label{drift-velocity}
        \end{align}
        \refe{drift-velocity}を\refe{eom-cyclotron}に代入すると
        \begin{align}
          m \dv{\bm{v}_{\r{c}}}{t} = q\bm{E} + q(\bm{v}_{\r{gc}} + \bm{v}_{\r{c}})\times\bm{B}
        \end{align}
        となる.ここで$\bm{v}_{\r{gc}}$が一定であることを用いた.
        整理すると,
        \begin{align}
          m \dv{\bm{v}_{\r{c}}}{t} = q(\bm{v}_{\r{c}} \times \bm{B}) + q\bm{E} + q(\bm{v}_{\r{gc}}\times\bm{B})
        \end{align}
        である.左辺と右辺の第1項はサイクロトロン運動,右辺の第2項と第3項は旋回中心の運動を表している.
        旋回中心の運動に注目すると,
        \begin{align}
          \bm{0} = q\bm{E} + q(\bm{v}_{\r{gc}}\times\bm{B})
        \end{align}
        この両辺を$q$で割り,$\bm{B}$を外積する.ベクトル解析の公式より
        \begin{align}
          B^2 \bm{v}_{\r{gc}} = -\bm{B} \times \bm{E}
        \end{align}
        が得られる.よって,
        \begin{align}
          \bm{v}_{\r{gc}} = \frac{\bm{E}\times\bm{B}}{B^2}
        \end{align}
        である.

      \subsubsection{一般的な力によるドリフト}
        $\bm{E}\times\bm{B}$ドリフトによる旋回中心の移動は,電場による力によって粒子が加速or減速され,そのため場所によってLarmor半径が変わることによるものであった.
        この考え方は他の場合にも適用できる.

        一般な力$\bm{F}$のときは
        \begin{align}
          q\bm{E} \to \bm{F}
        \end{align}
        とすればよく,ドリフト速度は
        \begin{align}
          \bm{v}_{F} = \frac{1}{q}\frac{\bm{F}\times\bm{B}}{B^2}
        \end{align}
        である.例えば重力の場合は
        \begin{align}
          \bm{v}_{F} = \frac{m}{q}\frac{\bm{g}\times\bm{B}}{B^2}
        \end{align}
        となる.
      
      \subsubsection{非一様磁場中でのドリフト}
        \paragraph{磁場勾配ドリフト}
          \textbf{磁場勾配ドリフト}(grad-$B$ drift motion)\index{じばこうばいどりふと@磁場勾配ドリフト}を考える.磁場が$z$方向で,磁場勾配が$x$方向にある場合を考える.
          以下にそれを見る.

          非一様磁場中ではLorentz力の大きさが位置ごとに変化する.磁場が大きいところでは旋回半径は大きく,磁場が小さいところでは旋回半径は小さくなる.
          よって,$\bm{B},\nabla B$の両方に垂直な方向にドリフト運動が生じる.また,イオンと電子はドリフトの縫合が逆なので電流が生じる.
          \textbf{磁場勾配ドリフト速度}(gradient-$B$ drift velocity)\index{じばこうばいどりふとそくど@磁場勾配ドリフト速度}は
          \begin{align}
            \bm{v}_{\nabla B} = \pm \frac{1}{2} v_{\perp} r_{\r{L}} \frac{\bm{B} \times \nabla B}{B^2}
          \end{align}
          である.

          以下に導出を行う.まず,粒子の運動方程式の各成分は
          \begin{align}
            \begin{dcases}
              m \dv{v_x}{t} &= qv_yB\\
              m \dv{v_y}{t} &= - v_xB\\
              m \dv{v_z}{t} &= 0
            \end{dcases}
          \end{align}
          である.ここで$y$方向について着目する.
          \begin{align}
            m \dv{v_y}{t} = -q v_x B
          \end{align}
          磁場を原点周りでTaylor展開する.
          \begin{align}
            \bm{B} &= \bm{B}_0 + (\bm{r} \cdot \nabla)\bm{B}\\
            B_z &= B_0 + y\pdv{B_z}{y}
          \end{align}
          ここで,初期条件より
          \begin{align}
            v_x &= v_{\perp} \cos(\omega_{\r{c}}t)\\
            y &= \frac{v_{\perp}}{\omega_{\r{c}}}\cos(\omega_{\r{c}}t)
          \end{align}
          である.よって,
          \begin{align}
            m \dv{v_y}{t} &= -qv_x B\\
            &=  -q v_{\perp} \cos(\omega_{\r{c}}t) (B_0 + \frac{v_{\perp}}{\omega_{\r{c}}}\cos(\omega_{\r{c}}t)\pdv{B_z}{y})\\
          \end{align}
          第1項は通常のLorentz力なので,第2項に着目する.

          一般的な力$\bm{F}$によるドリフトの式より,
          \begin{align}
            \bm{v}_{\r{gc}} &= \frac{\bm{F}\times\bm{B}}{B^2}\\
            &= \mp \frac{1}{2} \frac{r_{\r{L}}v_{\perp}}{B} \pdv{B_z}{y}\bm{e}_x\\
            &= \mp \frac{1}{2} \frac{v_{\perp} r_{\perp}}{B} \pdv{B_z}{y}\bm{e}_y \times \bm{e}_z\\
            &= \mp \frac{1}{2} v_{\perp} r_{\r{L}} \frac{\bm{B} \times \nabla B}{B^2}
          \end{align}
          が得られる.

        \paragraph{曲率ドリフト}
          \textbf{曲率ドリフト}(curvature drift)\index{きょくどりふと@曲率ドリフト}を考える.磁力線が湾曲整定る場合には,
          磁場勾配ドリフトに加えて,遠心力に起因する曲率ドリフトが生じる.

          粒子が感じる平均的な遠心力は
          \begin{align}
            \bm{F}_{\r{cf}} = \frac{m v_{\parallel}^2}{R_{\r{c}}}\bm{e}_r = m v_{\parallel}^2 \frac{\bm{R}_{\r{c}}}{R_{\r{c}}^2}
          \end{align}
          である.よって,一般的な力によるドリフト速度の式より,曲率ドリフト速度は
          \begin{align}
            \bm{v}_R = \frac{mv_{\parallel}^2}{qB^2}\frac{\bm{R}_{\r{c}} \times \bm{B}}{R_{\r{c}}^2}
          \end{align}
          となる.

          次に,磁場勾配を曲率を使って表す.
          電流$I$のまわりには
          \begin{align}
            B &= \frac{\mu_0 I}{2\pi R_{\r{c}}}\\
            \nabla B &\parallel \bm{e}_r
          \end{align}
          となる.よって,
          \begin{align}
            \frac{\nabla B}{B} = -\frac{\bm{R}_{\r{c}}}{R_{\r{c}}^2}
          \end{align}
          が得られる.したがって,磁場勾配ドリフトは,
          \begin{align}
            \bm{v}_{\nabla B} = \frac{1}{2}\frac{m}{q}v_{\perp}^2 \frac{\bm{R}_{\r{c}} \times \bm{B}}{R_{\r{c}}^2 B^2}
          \end{align}
          となる.

          以上より,曲率ドリフトと磁場勾配ドリフトの和である\textbf{磁気ドリフト}(magnetic drift)\index{じきどりふと@磁気ドリフト}は
          \begin{align}
            \bm{v}_{R} + \bm{v}_{\nabla B} = \frac{1}{2}\qty(v_{\parallel}^2 + \frac{1}{2}v_{\perp}^2)\frac{\bm{R}_{\r{c}} \times \bm{B}}{R_{\r{c}}^2 B^2}
          \end{align}
          である.磁場勾配で表すと,
          \begin{align}
            \bm{v}_{R} + \bm{v}_{\nabla B} =\frac{m}{qB}\qty(v_{\parallel}^2 + \frac{1}{2}v_{\perp}^2) \frac{\bm{B} \times \nabla B}{B^2}
          \end{align}
          となる.しかし,上の式を見てわかるように,電子とイオンでドリフトの向きが逆である.
          よって,荷電分離により電場が発生する.そのため,$\bm{E}\times\bm{B}$ドリフトが生じる.これは外側に向いている.したがって,プラズマ全体が損失してしまう.
          つまり,単純なトーラス磁場では,プラズマの閉じ込めは不完全である.

    \subsection{磁気鏡}
      \subsubsection{磁気鏡とは?}
        \textbf{磁気鏡}(magnetic mirror)\index{じきかがみ@磁気鏡}は,磁場が強いところに荷電粒子が近づくと,そこで反射される現象をいう.

      \subsubsection{磁気モーメント}
        磁気ミラー配位での粒子の運動を考えるための準備として,ここでは荷電粒子が有する磁気モーメント\index{じきもーめんと@磁気モーメント}について述べる.また,
        磁気モーメント及びその断熱不変性は,磁気ミラー配位に限らずいろいろな磁場配位での運動を考えるうえで重要となる.

        磁化$m$が長さ$L$離れて存在しているとき,磁気モーメント
        \begin{align}
          \mu = 2mL
        \end{align}
        である.
        円電流による磁気モーメントは
        \begin{align}
          \mu = I \times S
        \end{align}
        である.

        ここで,磁場$B$中の荷電粒子を考える.荷電粒子は半径$r_{\r{L}}$でサイクロトロン運動をしているので
        \begin{align}
          S = \pi \times r_{\r{L}}^2
        \end{align}
        である.また,電流は,旋回時間を$T_{\r{c}}$として
        \begin{align}
          I = e \times \qty(\frac{1}{T_{\r{c}}})
        \end{align}
        である.よって,荷電粒子のつくる磁気モーメントは
        \begin{align}
          \mu = IS = \frac{e\pi r_{\r{L}}^2}{T_{\r{c}}}
        \end{align}
        となる.さらに,
        \begin{align}
          r_{\r{L}} = v_{\perp} / \omega_{\r{c}},\ T_{\r{c}} = 2\pi / \omega_{\r{c}},\ \omega_{\r{c}} = eB/m
        \end{align}
        より,
        \begin{align}
          \mu = \frac{(1/2)mv_{\perp}^2}{B}
        \end{align}
        となる.もともとの磁場と反対方向であることに注意.反磁性である.

        磁場の空間的時間的変化が緩やかなとき,磁気モーメントは保存される\footnote{\refsss{sec:conservation-of-magnetic-moment}にて述べる.}.



      \subsubsection{磁気鏡による粒子反射の物理機構}

      \subsubsection{磁気モーメントの保存性の証明}
      \label{sec:conservation-of-magnetic-moment}
        粒子の全運動エネルギーは
        \begin{align}
          W &= \frac{1}{2}mv_{\parallel}^2 + \frac{1}{2}mv_{\perp}^2\\
          &= \frac{1}{2}mv_{\parallel}^2 + \mu B
        \end{align}
        と書ける.時間微分は
        \begin{align}
          \dv{W}{t} = mv_{\parallel}\dv{v_{\parallel}}{t} + \dv{\mu B}{t}
        \end{align}
        である.
\end{document}
