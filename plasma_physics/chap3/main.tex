\documentclass{report}
\input{../head.tex}
\begin{document}
  この章ではプラズマがどのような運動をするかを学ぶ.
  \section{はじめに}
    プラズマ輸送の解析には\textbf{流体モデル}(Fluid Model)\index{りゅうたいもでる@流体モデル}と\textbf{運動論的モデル}(Kinetic Model)\index{うんどうろんてきもでる@運動論的モデル}を用いる.
    前者はプラズマを連続体として扱い,巨視的なプラズマの振る舞いを記述する.後者は粒子個々の運動を解き,微視的なプラズマの振る舞いを記述する.
  \section{一様磁場中での粒子の運動}
    \subsection{磁場に垂直方向の運動}
      一様磁場中では電荷はサイクロトロン運動\index{さいくろとろんうんどう@サイクロトロン運動}をする.美束密度$\bm{B}$中を電荷$q$を持つ粒子が速度$\bm{v}$で運動しているとする.
      この粒子が受ける力は
      \begin{align}
        \bm{F} = q\bm{v}\times\bm{B}
      \end{align}
      である.ここで,
      \begin{align}
        \bm{F} \cdot \bm{v} = 0
      \end{align}
      なのでLorentz力は仕事をしないことがわかる.
      この粒子の運動方程式は以下のようになる.
      \begin{align}
        m \dv{\bm{v}}{t} = q\bm{v}\times\bm{B}
      \end{align}
      ここで,$\bm{B} = \qty(0, 0, B)$,$\bm{v} = \qty(v_x, v_y, v_z)$とする.運動方程式の各成分は
      \begin{align}
        \begin{dcases}
          m\dv{v_x}{t} = qv_yB \\
          m\dv{v_y}{t} = -qv_xB \\
          m\dv{v_z}{t} = 0
        \end{dcases}
      \end{align}
      となる.これらを整理すると,
      \begin{align}
        &\ddot{v}_x = \frac{qB}{m}v_x \to \ddot{v}_x = - \omega_{\r{c}}^2 v_x\\
        &\ddot{v}_y = -\frac{qB}{m}v_y \to \ddot{v}_y = \omega_{\r{c}}^2 v_y
      \end{align}
      となる.
      よって,荷電粒子は$xy$平面上を\textbf{サイクロトロン周波数}(cyclotron frequency)$\omega_{\r{c}} = qB/m$\index{さいくろとろんしゅうはすう@サイクロトロン周波数}で
      円運動することがわかる\footnote{
        通常,サイクロトロン周波数という場合には,電荷$q$の絶対値をとって,$\omega_{\r{c}} = |q|B/m$とすることが多い.
      }
      .また,初期条件
      \begin{align}
        \begin{dcases}
          x = 0\\
          y = 0\\
          v_x = v_{\perp}\\
          v_y = 0
        \end{dcases}
        \ \r{at}\ t = 0
      \end{align}
      の下で,この粒子の軌道は
      \begin{align}
        \begin{dcases}
          x(t) = \frac{v_{\perp}}{\omega_{\r{c}}}\sin(\omega_{\r{c}}t)\\
          y(t) = \frac{v_{\perp}}{\omega_{\r{c}}}(\cos(\omega_{\r{c}}t)-1)
        \end{dcases}
      \end{align}
      となる.したがって,
      \begin{align}
        x^2 + \qty(y + \frac{v_{\perp}}{\omega_{\r{c}}})^2 = \qty(\frac{v_{\perp}}{\omega_{\r{c}}})^2
      \end{align}
      が成り立つので,この粒子の円運動の\textbf{旋回中心}\index{せんかいちゅうしん@旋回中心}は
      \begin{align}
        (x_g,y_g) = (0, -\frac{v_{\perp}}{\omega_{\r{c}}})
      \end{align}
      で\textbf{旋回半径}\index{せんかいはんけい@旋回半径}は
      \begin{align}
        r_{\r{L}} = \frac{v_{\perp}}{\abs{\omega_{\r{c}}}}
      \end{align}
      である.

      まとめると,電荷数$Z$の粒子のサイクロトロン周波数は
      \begin{align}
        \omega_{\r{c}j} = \frac{Z e B}{m_j}
      \end{align}
      である\footnote{$j$はイオンか電子かを示す.}.サイクロトロン半径は
      \begin{align}
        r_{\r{L}j} = \frac{m_j v_{\perp}}{ZeB}
      \end{align}
      である.また,熱速度$v_{{\r{th}}j}$を用いて
      \begin{align}
        r_{\r{L}j} = \frac{m_j v_{{\r{th}}j}}{ZeB}
      \end{align}
      と表すことが多い.

      $z$軸方向には力は働かないので
      \begin{align}
        v_{\parallel} &= \dot{v_z} = \r{const.}\\
        z &= z_0 + v_{\parallel}t
      \end{align}
      である.つまり,全体としてはらせん運動をしている.

    \subsection{ドリフト運動}
      
      ここでは一様磁場に加えて電場や重力などが働く場合の粒子の運動や磁場が空間的に変化する場合の粒子の運動を考える.

      磁場が$z$軸方向に向いているとする.前節とり粒子は$xy$平面上を円運動するのであった.ここに電場を$x$方向に加える.
      このとき,粒子は一方では加速されもう一方では減速される.つまり,上方では$V_0+V_1$,下方では$V_0-V_1$のように異なってしまう.
      速度が$V_0+V_1$のときは旋回半径が大きく,$V_0-V_1$のときは旋回半径が小さくなる.よって,旋回中心が
      下方へドリフトしていく.
\end{document}
