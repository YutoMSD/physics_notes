\documentclass{report}
\input{../head}
\begin{document}
  \section{はじめに}
    本章ではプラズマの\textbf{流体モデル}(Fluid Model)\index{りゅうたいもでる@流体モデル}によるモデル化を進める.
    流体モデルは一般に以下の3つの方程式及びMaxwell方程式に基礎を置く.
    \begin{enumerate}
      \item 密度連続の式
      \item 運動方程式
      \item エネルギー方程式
    \end{enumerate}
    流体モデルはプラズマの集団的・巨視的な振る舞いの理解に役立つ.
  
  \section{密度連続の式}
    プラズマの存在する空間内に,閉曲面$S$を考え,その体積を$V$とする.体積中の粒子の総数$N$を考える.密度を$n$とする.
    粒子(数)密度の連続の式は
    \begin{align}
      \pdv{n}{t} + \div(n \bm{u}) = 0
    \end{align}
    である.ここで,$\bm{u}$はプラズマの流れの平均速度ベクトルである.粒子の生成あるは消滅がある場合は生成消滅項$S_i$を右辺に加える\footnote{$S_i>0$で生成.$S_i<0$で消滅.}.
    \begin{align}
      \pdv{n}{t} + \div(n \bm{u}) = S_i
    \end{align}
    また,質量密度の連続の式は
    \begin{align}
      \pdv{\rho}{t} + \div(\rho \bm{u}) = 0
    \end{align}
    である.

    場の考え方に慣れるために,波形を保ったまま,一定速度で海岸に打ち寄せる波を例に考える.
    簡単のために1次元とし$\rho(x,t)$,流体の速度は空間的,時間的に変化しないとする$u_x = \r{const.}$\footnote{一般的には空間と時間に依存する.}.
    これらを粒子密度の連続の式に代入すると,
    \begin{align}
      \pdv{\rho}{t} + u_x \pdv{\rho}{x} = 0 \label{eq:cont}
    \end{align}
    が得られる.\refe{eq:cont}の解は
    \begin{align}
      \rho(x,t) = \rho(x-u_x t) = \r{const.}
    \end{align}
    である.これは波形を保ったまま,$x$方向に伝わる波を表している.\reff{eq:cont}の第1項は固定した位置における密度の時間変化,第2項は流体の対流による密度の変化を表している.
    つまり,生成・消滅がないとき,空間の各点での密度変化は流体の対流による流入・流出によって生じる.

    また,波の頂点にいる観測者から見ると,
    \begin{align}
      \dv{\rho}{t} = 0
    \end{align}
    なので,
    \begin{align}
      \dv{\rho}{t} = \pdv{\rho}{t} + u_x \pdv{\rho}{x} = 0
    \end{align}
    つまり,
    \begin{align}
      \qty[\pdv{t} + u_x \pdv{x}]\rho = 0
    \end{align}
    である.この,
    \begin{align}
      \dv{t} = \pdv{t} + u_x \pdv{x}
    \end{align}
    を\textbf{対流微分}(Convective Derivative)\index{たいりゅうびぶん@対流微分}という.

    より一般的には,
    \begin{align}
      \pdv{\rho}{t} + \div(\rho \bm{u}) = 0
    \end{align}
    である.ここで,非圧縮性流体$\div \bm{u} = 0$を考えると,
    \begin{align}
      \div(\rho \bm{u}) = \bm{u} \cdot \nabla \rho
    \end{align}
    よって,
    \begin{align}
      \pdv{\rho}{t} + \bm{u} \cdot \nabla \rho = 0
    \end{align}
    である.

    また,次節では以下の\textbf{Euler方程式}\index{Eulerほうていしき@Euler方程式}を導出する.
    \begin{align}
      mn\qty[\pdv{t} + (\bm{u} \cdot \nabla)]\bm{u} = nq(\bm{E} + \bm{u} \times \bm{B}) - \nabla p
    \end{align}
    また,対流微分は
    \begin{align}
      \frac{\r{D}}{\r{D}t} = \pdv{t} + \bm{u} \cdot \nabla
    \end{align}
    とも書かれる.

  \section{流体の運動方程式}
    粒子1個当たりの運動量は$m\bm{u}$である.単位体積当たりの運動量は$n m \bm{u}$である.ここで,粒子の質量を$m$,粒子の数密度を$n$とする.
    体積$V$に含まれる粒子が持つ全体の運動量$\bm{M}$は$\bm{M} = \int_V n m \bm{u} \dd{V}$である.

    $\bm{M}$の時間変化を考える.時間変化に寄与するのは,(1)外力による運動量の変化(2)面$S$を通しての流体の流入・流出(3)面$S$を介して,面の外側の流体が及ぼす圧力,がある.

    \subsection{外力による運動量の時間変化}
      位置$\bm{r}$時刻$t$において流体粒子1個あたりに働く力を$\bm{f}^{\r{ext}}(\bm{r},t)$とする\footnote{例:$\bm{f}^{\r{ext}}(\bm{r},t) = q\qty[\bm{E} + \bm{v} \times \bm{B}]$}.
      体積$V$全体に働く力は,流体密度を$n(\bm{r},t)$とすると,$\int_V n(\bm{r},t) \bm{f}^{\r{ext}}(\bm{r},t) \dd{V}$である.
      よって,$\bm{M}$の時間変化は,
      \begin{align}
        \pdv{\bm{M}}{t} = \int_V \pdv{t}(n m \bm{u}) \dd{V} = \int_V n(\bm{r},t) \bm{f}^{\r{ext}}(\bm{r},t) \dd{V}
      \end{align}
      である.したがって,この式が恒等的に成り立つのは
      \begin{align}
        \pdv{t}(n m \bm{u}) = n \bm{f}^{\r{ext}}
      \end{align}
      が満たされるときである.密度が時間的に変化しない時,
      \begin{align}
        \pdv{t}(m \bm{u}(\bm{u},t)) = \bm{f}^{\r{ext}}(\bm{r},t)
      \end{align}
      である.

    \subsection{運動量の流入・流出}
      体積$V$中の粒子総数$N$の時間変化を考える.
      \begin{align}
        \pdv{N}{t} = \int_S n\bm{u} \cdot \dd{\bm{S}}
      \end{align}
      である.ここで,$\bm{\Gamma} = n\bm{u}$を\textbf{粒子束密度}(Particle Flux Density)\index{りゅうしそくみつど@粒子束密度}という.
      表面を通しての運動量の流入・流出は,$(m\bm{u})(n\bm{u}\cdot \dd{\bm{S}})$である.
      したがって,$\bm{M}$の時間変化は,
      \begin{align}
        \pdv{\bm{M}}{t} = -\int_S (m\bm{u})(n\bm{u}\cdot \dd{\bm{S}})
      \end{align}
      である.左辺は$\int_V \pdv{t}(n m \bm{u}) \dd{V}$と書けるので,$x$成分で考えると,Gaussの定理を使うと,
      \begin{align}
        \int_V \qty[\pdv{t}(mnu_x) + \nabla \cdot (mnu_x \bm{u})] \dd{V} = 0
      \end{align}
      である.よって,各成分は
      \begin{align}
        \pdv{t}(mnu_x) + \nabla \cdot (mnu_x \bm{u}) = 0\\
        \pdv{t}(mnu_y) + \nabla \cdot (mnu_y \bm{u}) = 0\\
        \pdv{t}(mnu_z) + \nabla \cdot (mnu_z \bm{u}) = 0
      \end{align}
      である.ここで$\nabla \cdot (mnu_x \bm{u}) $の物理的意味を考察する.
      右辺に移項すると,
      \begin{align}
        \pdv{t}(mnu_x) = -\nabla \cdot (mnu_x \bm{u})
      \end{align}
      と書けるので,$-\nabla \cdot (mnu_x \bm{u})$は,流体の単位体積当たりに働く力に相当する.
      さらにベクトル解析の公式を用いて展開すると,
      \begin{align}
        \nabla \cdot (mnu_x \bm{u}) = \pdv{x}(mnu_x u_x) + \pdv{y}(mnu_x u_y) + \pdv{z}(mnu_x u_z)
      \end{align}
      となる.ここで,\textbf{運動量流速密度}(Momentum Flux Density)\index{うんどうりょうりゅうそくみつど@運動量流速密度}を$P_{ij} = mnu_i u_j$を用いると,
      \begin{align}
        \nabla \cdot (mnu_x \bm{u}) = \pdv{x}P_{xx} + \pdv{y}P_{xy} + \pdv{z}P_{xz}
      \end{align}
      と書ける\footnote{$P_{xx} = mnu_xu_x = (mu_x)(nu_x)$と書けるので,これは単位時間に$x$軸に垂直な単位面積を横切る$x$方向の運動量を表している.また,$\pdv{P_{xx}}{x}\ [\frac{\r{N}}{\r{m}^3}]$は流体の単位体積当たりに働く力に相当する.}.

      次に,$P_{xy},P_{xz},\pdv{P_{xy}}{y},\pdv{P_{xz}}{z}$の物理的意味を考えよう.
\end{document}