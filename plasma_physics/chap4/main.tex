\documentclass{report}
\usepackage{luatexja}
\usepackage{amsmath, amssymb, type1cm, amsfonts, latexsym, mathtools, bm, amsthm, url, color}
\usepackage{multirow, hyperref, longtable, dcolumn, tablefootnote, physics}
\usepackage{tabularx, footmisc, colortbl, here, usebib, microtype}
\usepackage{graphicx, luacode, caption, fancyhdr}
\usepackage[top = 20truemm, bottom = 20truemm, left = 20truemm, right = 20truemm]{geometry}
\usepackage{ascmac, fancybox, color, tabularray, subcaption}
\usepackage{luatexja-fontspec, multicol}
\usepackage{upgreek, colortbl, mhchem}
\usepackage{biblatex, array, truthtable}
\usepackage{listings, jvlisting}
\usepackage{xcolor, siunitx, float, dcolumn}
\sisetup{
  table-format = 1.5, % 小数点以下の桁数を指定
  table-number-alignment = center, % 数値の中央揃え
}
% \abovedisplayskip = 0pt
% \belowdisplayskip = 0pt
\allowdisplaybreaks
% \DeclarePairedDelimiter{\abs}{\lvert}{\rvert}
\newcolumntype{t}{!{\vrule width 0.1pt}}
\newcolumntype{b}{!{\vrule width 1.5pt}}
\UseTblrLibrary{amsmath, booktabs, counter, diagbox, functional, hook, html, nameref, siunitx, varwidth, zref}
\setlength{\columnseprule}{0.4pt}
\captionsetup[figure]{font = bf}
\captionsetup[table]{font = bf}
\captionsetup[lstlisting]{font = bf}
\captionsetup[subfigure]{font = bf, labelformat = simple}
\setcounter{secnumdepth}{5}
\newcolumntype{d}{D{.}{.}{5}}
\newcolumntype{M}[1]{>{\centering\arraybackslash}m{#1}}
\everymath{\displaystyle}
\DeclareMathOperator*{\AND}{\cdot}
\DeclareMathOperator*{\NAND}{NAND}
% \DeclareMathOperator*{\NOT}{NOT}
\DeclareMathOperator*{\OR}{+}
% \let\oldbar\bar
\renewcommand{\i}{\mathrm{i}}
\renewcommand{\laplacian}{\Delta}
\newcommand{\NOT}[1]{\overline{#1}}
\renewcommand{\hat}[1]{\overhat{#1}}
\renewcommand{\thesubfigure}{(\alph{subfigure})}
\newcommand{\m}[3]{\multicolumn{#1}{#2}{#3}}
\renewcommand{\r}[1]{\mathrm{#1}}
\newcommand{\e}{\mathrm{e}}
\newcommand{\Ef}{E_{\mathrm{F}}}
\renewcommand{\c}{\si{\degreeCelsius}}
\renewcommand{\d}{\r{d}}
\renewcommand{\t}[1]{\texttt{#1}}
\newcommand{\kb}{k_{\mathrm{B}}}
\renewcommand{\phi}{\varphi}
% \newcommand{\dv}[3]{\frac{\d #1}{\d #2}}
% \newcommand{\pdv}[2]{\frac{\partial #1}{\partial #2}}
% \newcommand{\qtys}[#1]{\left(#1 \right)}
% \newcommand{\qtym}[#1]{\left\{#1\right\}}
% \newcommand{\qtyl}[#1]{\left[#1\right]}
\newcommand{\reff}[1]{\textbf{図\ref{#1}}}
\newcommand{\reft}[1]{\textbf{表\ref{#1}}}
\newcommand{\refe}[1]{\textbf{式\eqref{#1}}}
\newcommand{\refp}[1]{\textbf{コード\ref{#1}}}
\newcommand{\refa}[1]{\textbf{\ref{#1}}}
\renewcommand{\lstlistingname}{コード}
\renewcommand{\theequation}{\thesection.\arabic{equation}}
\renewcommand{\footrulewidth}{0.4pt}
\newcommand{\mar}[1]{\textcircled{\scriptsize #1}}
\newcommand{\combination}[2]{{}_{#1} \mathrm{C}_{#2}}
\newcommand{\thline}{\noalign{\hrule height 0.1pt}}
\newcommand{\bhline}{\noalign{\hrule height 1.5pt}}
\newcommand*{\myCurrentTime}{
  \directlua{ my_current_time() }
}
\newcommand{\Rnum}[1]{
  \ifnum #1 = 1
    I
  \fi
  \ifnum #1 = 2
    I\hspace{-1.2pt}I
  \fi
  \ifnum #1 = 3
    I\hspace{-1.2pt}I\hspace{-1.2pt}I
  \fi
  \ifnum #1 = 4
    I\hspace{-1.2pt}V
  \fi
  \ifnum #1 = 5
    V
  \fi
  \ifnum #1 = 6
    V\hspace{-1.2pt}I
  \fi
  \ifnum #1 = 7
    V\hspace{-1.2pt}I\hspace{-1.2pt}I
  \fi
  \ifnum #1 = 8
    V\hspace{-1.2pt}I\hspace{-1.2pt}I\hspace{-1.2pt}I
  \fi
  \ifnum #1 = 9
    I\hspace{-1.2pt}X
  \fi
  \ifnum #1 = 10
    X
  \fi
}
\newcommand{\cover}{
  \renewcommand{\arraystretch}{3}
  \title{物理情報工学実験報告書}
  \date{}
  \author{}
  \maketitle
  \begin{table}[H]
    \begin{flushright}
      2024年度
    \end{flushright}
    \begin{center}
      \begin{tabularx}{150mm}{|>{\centering}p{40mm}|>{\centering}p{25mm}|>{\centering}p{30mm}|>{\centering\arraybackslash}X|}
        \hline
        \Large{実験テーマ} & \multicolumn{3}{c|}{\Large{A1(直流安定化電源)}} \\ \hline
        \Large{担当教員名} & \multicolumn{3}{c|}{\Large{塚田孝祐}} \\ \hline
        \Large{実験整理番号} & \Large{002} & \Large{実験者氏名} & \Large{青木\ 陽}\\ \hline
        \Large{共同実験者氏名} & \multicolumn{3}{c|}{} \\ \hline
        \Large{曜日組} & \Large{火1班} & \Large{実験日} & \Large{6月25日} \\ \hline
        \Large{実験回} & \Large{9} & \Large{報告書提出日} & \Large{\myCurrentTime}\\ \hline
      \end{tabularx}
    \end{center}
  \end{table}
  \thispagestyle{empty} 
  \addtocounter{page}{-1}
  \clearpage
  \renewcommand{\arraystretch}{1.0}
}
\pagestyle{fancy}
\chead{物性物理II}
\rhead{}
\cfoot{\thepage}
\lhead{}
\rfoot{\t{harry\_arbrebleu}}
\setcounter{tocdepth}{4}
\makeatletter
\@addtoreset{equation}{subsection}
\makeatother
\begin{luacode*}
  function my_current_time()
    local date = os.date("*t")
    local year = date.year
    local month = date.month
    local day = date.day
    local hour = date.hour
    local min = date.min
    local sec = date.sec
    local formatted_date = string.format("%d月%d日", month, day)
    tex.sprint(formatted_date)
  end
\end{luacode*}
\lstset{
  language = Matlab, % Set the language for the code
  basicstyle = {\ttfamily},
  identifierstyle = {\small},
  commentstyle = \color{red},
  keywordstyle = \color{blue},
  ndkeywordstyle = {\small},
  stringstyle = \color{orange},
  frame={tb},
  breaklines = true,
  columns=[l]{fullflexible},
  xrightmargin = 5mm,
  xleftmargin = 5mm,
  numberstyle = {\ttfamily\scriptsize},
  stepnumber = 1,
  numbersep = 1mm,
  lineskip = -0.5ex,
  showstringspaces = false,
  numbers = left,
  frame = lines,
  backgroundcolor = \color{gray!10},
  rulecolor = \color{black!30},
}

\definecolor{mygray}{rgb}{0.5,0.5,0.5}
\definecolor{mymauve}{rgb}{0.58,0,0.82}
\definecolor{mygreen}{rgb}{0,0.6,0}

\lstset{ %
  backgroundcolor=\color{white},   % 背景色
  basicstyle=\ttfamily\footnotesize, % 基本の書体スタイル
  breakatwhitespace=false,        % 空白で行分割しない
  breaklines=true,                % 長い行は分割する
  captionpos=b,                   % キャプションの位置
  commentstyle=\color{mygreen},   % コメントのスタイル
  extendedchars=true,             % 非 ASCII 文字をサポート
  frame=single,                   % フレームの表示
  keywordstyle=\color{blue},      % キーワードのスタイル
  language=[LaTeX]TeX,            % 言語を LaTeX に設定
  numbers=left,                   % 行番号を左側に表示
  numbersep=5pt,                  % 行番号とコードの間の距離
  numberstyle=\tiny\color{mygray}, % 行番号のスタイル
  rulecolor=\color{black},        % 枠線の色
  showspaces=false,               % スペースを表示しない
  showstringspaces=false,         % 文字列内のスペースを表示しない
  showtabs=false,                 % タブを表示しない
  stepnumber=1,                   % 行番号を表示する間隔
  stringstyle=\color{mymauve},    % 文字列のスタイル
  tabsize=2,                      % タブの幅
  title=\lstname                  % タイトル
}
\lstset{
  language = C++, % Set the language for the code
  basicstyle = {\ttfamily},
  identifierstyle = {\small},
  commentstyle = \color{red},
  keywordstyle = \color{blue},
  ndkeywordstyle = {\small},
  stringstyle = \color{orange},
  frame={tb},
  breaklines = true,
  columns=[l]{fullflexible},
  xrightmargin = 5mm,
  xleftmargin = 5mm,
  numberstyle = {\ttfamily\scriptsize},
  stepnumber = 1,
  numbersep = 1mm,
  lineskip = -0.5ex,
  showstringspaces = false,
  numbers = left,
  frame = lines,
  backgroundcolor = \color{gray!10},
  rulecolor = \color{black!30},
}
\title{物性物理II}
\date{\today}
\author{\t{harry\_arbrebleu}}
\addbibresource{ref.bib}
\defbibheading{bunken}[\refname]{\section*{#1}}
\begin{document}
  \section{はじめに}
    本章ではプラズマの\textbf{流体モデル}(Fluid Model)\index{りゅうたいもでる@流体モデル}によるモデル化を進める.
    流体モデルは一般に以下の3つの方程式及びMaxwell方程式に基礎を置く.
    \begin{enumerate}
      \item 密度連続の式
      \item 運動方程式
      \item エネルギー方程式
    \end{enumerate}
    流体モデルはプラズマの集団的・巨視的な振る舞いの理解に役立つ.
  
  \section{密度連続の式}
    プラズマの存在する空間内に,閉曲面$S$を考え,その体積を$V$とする.体積中の粒子の総数$N$を考える.密度を$n$とする.
    粒子(数)密度の連続の式は
    \begin{align}
      \pdv{n}{t} + \div(n \bm{u}) = 0
    \end{align}
    である.ここで,$\bm{u}$はプラズマの流れの平均速度ベクトルである.粒子の生成あるは消滅がある場合は生成消滅項$S_i$を右辺に加える\footnote{$S_i>0$で生成.$S_i<0$で消滅.}.
    \begin{align}
      \pdv{n}{t} + \div(n \bm{u}) = S_i
    \end{align}
    また,質量密度の連続の式は
    \begin{align}
      \pdv{\rho}{t} + \div(\rho \bm{u}) = 0
    \end{align}
    である.

    場の考え方に慣れるために,波形を保ったまま,一定速度で海岸に打ち寄せる波を例に考える.
    簡単のために1次元とし$\rho(x,t)$,流体の速度は空間的,時間的に変化しないとする$u_x = \r{const.}$\footnote{一般的には空間と時間に依存する.}.
    これらを粒子密度の連続の式に代入すると,
    \begin{align}
      \pdv{\rho}{t} + u_x \pdv{\rho}{x} = 0 \label{eq:cont}
    \end{align}
    が得られる.\refe{eq:cont}の解は
    \begin{align}
      \rho(x,t) = \rho(x-u_x t) = \r{const.}
    \end{align}
    である.これは波形を保ったまま,$x$方向に伝わる波を表している.\reff{eq:cont}の第1項は固定した位置における密度の時間変化,第2項は流体の対流による密度の変化を表している.
    つまり,生成・消滅がないとき,空間の各点での密度変化は流体の対流による流入・流出によって生じる.

    また,波の頂点にいる観測者から見ると,
    \begin{align}
      \dv{\rho}{t} = 0
    \end{align}
    なので,
    \begin{align}
      \dv{\rho}{t} = \pdv{\rho}{t} + u_x \pdv{\rho}{x} = 0
    \end{align}
    つまり,
    \begin{align}
      \qty[\pdv{t} + u_x \pdv{x}]\rho = 0
    \end{align}
    である.この,
    \begin{align}
      \dv{t} = \pdv{t} + u_x \pdv{x}
    \end{align}
    を\textbf{対流微分}(Convective Derivative)\index{たいりゅうびぶん@対流微分}という.

    より一般的には,
    \begin{align}
      \pdv{\rho}{t} + \div(\rho \bm{u}) = 0
    \end{align}
    である.ここで,非圧縮性流体$\div \bm{u} = 0$を考えると,
    \begin{align}
      \div(\rho \bm{u}) = \bm{u} \cdot \nabla \rho
    \end{align}
    よって,
    \begin{align}
      \pdv{\rho}{t} + \bm{u} \cdot \nabla \rho = 0
    \end{align}
    である.

    また,次節では以下の\textbf{Euler方程式}\index{Eulerほうていしき@Euler方程式}を導出する.
    \begin{align}
      mn\qty[\pdv{t} + (\bm{u} \cdot \nabla)]\bm{u} = nq(\bm{E} + \bm{u} \times \bm{B}) - \nabla p
    \end{align}
    また,対流微分は
    \begin{align}
      \frac{\r{D}}{\r{D}t} = \pdv{t} + \bm{u} \cdot \nabla
    \end{align}
    とも書かれる.

  \section{流体の運動方程式}
    粒子1個当たりの運動量は$m\bm{u}$である.単位体積当たりの運動量は$n m \bm{u}$である.ここで,粒子の質量を$m$,粒子の数密度を$n$とする.
    体積$V$に含まれる粒子が持つ全体の運動量$\bm{M}$は$\bm{M} = \int_V n m \bm{u} \dd{V}$である.

    $\bm{M}$の時間変化を考える.時間変化に寄与するのは,(1)外力による運動量の変化(2)面$S$を通しての流体の流入・流出(3)面$S$を介して,面の外側の流体が及ぼす圧力,がある.

    \subsection{外力による運動量の時間変化}
      位置$\bm{r}$時刻$t$において流体粒子1個あたりに働く力を$\bm{f}^{\r{ext}}(\bm{r},t)$とする\footnote{例:$\bm{f}^{\r{ext}}(\bm{r},t) = q\qty[\bm{E} + \bm{v} \times \bm{B}]$}.
      体積$V$全体に働く力は,流体密度を$n(\bm{r},t)$とすると,$\int_V n(\bm{r},t) \bm{f}^{\r{ext}}(\bm{r},t) \dd{V}$である.
      よって,$\bm{M}$の時間変化は,
      \begin{align}
        \pdv{\bm{M}}{t} = \int_V \pdv{t}(n m \bm{u}) \dd{V} = \int_V n(\bm{r},t) \bm{f}^{\r{ext}}(\bm{r},t) \dd{V}
      \end{align}
      である.したがって,この式が恒等的に成り立つのは
      \begin{align}
        \pdv{t}(n m \bm{u}) = n \bm{f}^{\r{ext}}
      \end{align}
      が満たされるときである.密度が時間的に変化しない時,
      \begin{align}
        \pdv{t}(m \bm{u}(\bm{u},t)) = \bm{f}^{\r{ext}}(\bm{r},t)
      \end{align}
      である.

    \subsection{運動量の流入・流出}
      \label{subsec:momentum_flux}
      体積$V$中の粒子総数$N$の時間変化を考える.
      \begin{align}
        \pdv{N}{t} = \int_S n\bm{u} \cdot \dd{\bm{S}}
      \end{align}
      である.ここで,$\bm{\Gamma} = n\bm{u}$を\textbf{粒子束密度}(Particle Flux Density)\index{りゅうしそくみつど@粒子束密度}という.
      表面を通しての運動量の流入・流出は,$(m\bm{u})(n\bm{u}\cdot \dd{\bm{S}})$である.
      したがって,$\bm{M}$の時間変化は,
      \begin{align}
        \pdv{\bm{M}}{t} = -\int_S (m\bm{u})(n\bm{u}\cdot \dd{\bm{S}})
      \end{align}
      である.左辺は$\int_V \pdv{t}(n m \bm{u}) \dd{V}$と書けるので,$x$成分で考えると,Gaussの定理を使うと,
      \begin{align}
        \int_V \qty[\pdv{t}(mnu_x) + \nabla \cdot (mnu_x \bm{u})] \dd{V} = 0
      \end{align}
      である.よって,各成分は
      \begin{align}
        \pdv{t}(mnu_x) + \nabla \cdot (mnu_x \bm{u}) = 0\\
        \pdv{t}(mnu_y) + \nabla \cdot (mnu_y \bm{u}) = 0\\
        \pdv{t}(mnu_z) + \nabla \cdot (mnu_z \bm{u}) = 0
      \end{align}
      である.ここで$\nabla \cdot (mnu_x \bm{u}) $の物理的意味を考察する.
      右辺に移項すると,
      \begin{align}
        \pdv{t}(mnu_x) = -\nabla \cdot (mnu_x \bm{u})
      \end{align}
      と書けるので,$-\nabla \cdot (mnu_x \bm{u})$は,流体の単位体積当たりに働く力に相当する.
      さらにベクトル解析の公式を用いて展開すると,
      \begin{align}
        \nabla \cdot (mnu_x \bm{u}) = \pdv{x}(mnu_x u_x) + \pdv{y}(mnu_x u_y) + \pdv{z}(mnu_x u_z)
      \end{align}
      となる.ここで,\textbf{運動量流速密度}(Momentum Flux Density)\index{うんどうりょうりゅうそくみつど@運動量流速密度}を$P_{ij} = mnu_i u_j$を用いると,
      \begin{align}
        \nabla \cdot (mnu_x \bm{u}) = \pdv{x}P_{xx} + \pdv{y}P_{xy} + \pdv{z}P_{xz}
      \end{align}
      と書ける\footnote{$P_{xx} = mnu_xu_x = (mu_x)(nu_x)$と書けるので,これは単位時間に$x$軸に垂直な単位面積を横切る$x$方向の運動量を表している.また,$\pdv{P_{xx}}{x}\ [\frac{\r{N}}{\r{m}^3}]$は流体の単位体積当たりに働く力に相当する.}.
      $-\pdv{P_{xx}}{x}$は外力と等価な働きを持つ.

      次に,$P_{xy},P_{xz},\pdv{P_{xy}}{y},\pdv{P_{xz}}{z}$の物理的意味を考えよう.
      \begin{align}
        P_{xy} = mnu_xu_y = (mu_x)(nu_y)
      \end{align}
      と書ける.$mu_x$は粒子1個が持つ$x$方向の運動量.$nu_y$は単位時間に,$y$軸に垂直な単位面積を横切る粒子の個数である.よって,$P_{xy}$は単位時間に,$y$軸に垂直な単位面積を横切る$x$方向の運動量である.同様に,
      $P_{xz}$は単位時間に,$z$軸に垂直な単位面積を横切る$x$方向の運動量である.

      以上を踏まえると,運動量の流入・流出は次のように表現できる.
      \begin{align}
        \pdv{t} mn
        \begin{pmatrix}
          u_x\\
          u_y\\
          u_z
        \end{pmatrix}
        +
        \begin{pmatrix}
          \pdv{x} & \pdv{y} & \pdv{z}
        \end{pmatrix}
        \begin{pmatrix}
          P_{xx} & P_{xy} & P_{xz}\\
          P_{yx} & P_{yy} & P_{yz}\\
          P_{zx} & P_{zy} & P_{zz}
        \end{pmatrix}
        = 0
      \end{align}
      つまり,
      \begin{align}
        \pdv{t} mn\bm{u} + \nabla \overset{\leftrightarrow}{P} = 0
      \end{align}
      である.ここで,$\overset{\leftrightarrow}{P}$は運動量流速密度を成分とするテンソルである.
      また,$\overset{\leftrightarrow}{P}$は対称テンソルであり,$P_{ij} = P_{ji}$を満たす.

      ところで,粒子数の保存則は
      \begin{align}
        \pdv{n}{t} + \nabla \cdot \bm{\Gamma} = 0 \label{eq:continuity}
      \end{align}
      と書けるのであった.運動量の流入・流出の式は
      \begin{align}
        \pdv{t} mn\bm{u} + \nabla \overset{\leftrightarrow}{P} = 0 \label{eq:momentum}
      \end{align}
      である.よって\refe{eq:continuity}と\refe{eq:momentum}を比較すると,\refe{eq:momentum}は流体の運動量密度の保存則を表していることがわかる.

      (1)外力によって生じる運動量の時間変化と(2)運動量の流入・流出によって生じる運動量の時間変化の両方が存在する場合についてまとめると
      \begin{align}
        \pdv{t}(mn\bm{u}) + \nabla \overset{\leftrightarrow}{P} = \bm{F}^{\r{ext}} = n\bm{f}^{\r{ext}} \label{eq:momentum2}
      \end{align}
      となる.

    \subsection{圧力勾配による力}
      辺が$\Delta x, \Delta y, \Delta z$の立方体を考える.この微小体積に働く$x$方向の正味の力は,圧力を$p$として,
      \begin{align}
        -\pdv{p}{x}\Delta x \Delta y \Delta z \bm{i}
      \end{align}
      である.同様に,$y,z$方向についても,
      \begin{align}
        -\pdv{p}{y}\Delta x \Delta y \Delta z \bm{j}\\
        -\pdv{p}{z}\Delta x \Delta y \Delta z \bm{k}
      \end{align}
      となる.よって,立方体全体に働く力は,
      \begin{align}
        - \qty[\pdv{p}{x}\bm{i} + \pdv{p}{y}\bm{j} + \pdv{p}{z}\bm{k}]\Delta x \Delta y \Delta z
      \end{align}
      である.$\Delta V = \Delta x \Delta y \Delta z$とすると,$-\nabla p \Delta V$と書けるので,単位体積当たりに働く力は
      \begin{align}
        \bm{F}^{\r{p}} = -\nabla p
      \end{align}
      である.以上で求めた力は流体の各点$(x,y,z)$に単位体積当たりに働く力と考えることができる.よって,\refe{eq:momentum2}は
      \begin{align}
        \pdv{t}(mn\bm{u}) + \nabla \overset{\leftrightarrow}{P} = \bm{F}^{\r{ext}} + \bm{F}^{\r{p}} \label{eq:momentum3}
      \end{align}
    と書ける\footnote{\refa{chap:appendix}にて,圧力勾配による力の微視的起源を運動論に基づき考察する.}.

  \subsection{運動方程式の保存形}
    \refe{eq:momentum3}は,(1)外力(2)運動量の流入・流出(3)圧力の全てを考慮した運動量密度の時間変化を表す式である.
    \begin{align}
      \pdv{t}(mn\bm{u}) = \bm{F}^{\r{ext}} - \nabla \overset{\leftrightarrow}{P} - \nabla p
    \end{align}
    ここで,
    \begin{align}
      \overset{\leftrightarrow}{p} = 
      \begin{pmatrix}
        p & 0 & 0\\
        0 & p & 0\\
        0 & 0 & p
      \end{pmatrix}
    \end{align}
    を定義すると,
    \begin{align}
      \nabla p = \begin{pmatrix}
        \pdv{p}{x}&
        \pdv{p}{y}&
        \pdv{p}{z}
      \end{pmatrix}
      \begin{pmatrix}
        p & 0 & 0\\
        0 & p & 0\\
        0 & 0 & p
      \end{pmatrix}
    \end{align}
    と書ける.よって,
    \begin{align}
      \pdv{t}(mn\bm{u}) = \bm{F}^{\r{ext}} - \nabla \overset{\leftrightarrow}{P} - \nabla \overset{\leftrightarrow}{p} = \bm{F}^{\r{ext}} - \nabla \overset{\leftrightarrow}{\Pi}
    \end{align}
    が得られる.これを整理した,
    \begin{align}
      \pdv{t}(mn\bm{u}) + \nabla \overset{\leftrightarrow}{\Pi} = \bm{F}^{\r{ext}}
    \end{align}
    が\textbf{保存系の運動方程式}\index{ほぞんけいのうんどうほうていしき@保存系の運動方程式}である.
\end{document}