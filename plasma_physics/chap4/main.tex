\documentclass{report}
\input{../head}
\begin{document}
  \section{はじめに}
    本章ではプラズマの\textbf{流体モデル}(Fluid Model)\index{りゅうたいもでる@流体モデル}によるモデル化を進める.
    流体モデルは一般に以下の3つの方程式及びMaxwell方程式に基礎を置く.
    \begin{enumerate}
      \item 密度連続の式
      \item 運動方程式
      \item エネルギー方程式
    \end{enumerate}
    流体モデルはプラズマの集団的・巨視的な振る舞いの理解に役立つ.
  
  \section{密度連続の式}
    プラズマの存在する空間内に,閉曲面$S$を考え,その体積を$V$とする.体積中の粒子の総数$N$を考える.密度を$n$とする.
    粒子(数)密度の連続の式は
    \begin{align}
      \pdv{n}{t} + \div(n \bm{u}) = 0
    \end{align}
    である.ここで,$\bm{u}$はプラズマの流れの平均速度ベクトルである.粒子の生成あるは消滅がある場合は生成消滅項$S_i$を右辺に加える\footnote{$S_i>0$で生成.$S_i<0$で消滅.}.
    \begin{align}
      \pdv{n}{t} + \div(n \bm{u}) = S_i
    \end{align}
    また,質量密度の連続の式は
    \begin{align}
      \pdv{\rho}{t} + \div(\rho \bm{u}) = 0
    \end{align}
    である.

\end{document}