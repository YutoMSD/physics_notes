\documentclass{report}
\input{../head.tex}
\begin{document}
  \section{プラズマとは?}
    物質に熱を加えていくと,固体,液体,気体へと変化する.
    気体にさらに熱を加え,原子がイオンと電子に分離する.
    この状態を\textbf{プラズマ}という.
    \begin{itembox}[l]{プラズマの定義}
      プラズマとは,荷電粒子を含んだほぼ中性の粒子集団.
    \end{itembox}
    ここで,粒子集団とは,\textbf{電離したイオンと電子を含む気体}のことである.
    プラズマと普通の中性気体との違いは以下のようになる.
    \begin{enumerate}
      \item 中性気体の相互作用は衝突を通してのみである.
      \item プラズマは荷電粒子なので,相互作用はCoulomb力である.これは多数の粒子間で遠距離まで及ぶ.
    \end{enumerate}
    上記で「ほぼ中性」と述べたのは,プラズマには時間的空間的電荷の偏りがあるからである.
    しかし,プラズマにはこの偏りを打ち消す性質(Debye遮蔽,プラズマ振動)がある.
    ラフに言えばこうである.まず,ある正電荷の周りに電子が集まる.この領域は電子が増えるので電気的に負になる.
    ほかの領域は電気的に正になる.電子はこの正の領域に戻っていく.
    プラズマ物理ではエネルギーや温度が指標に使われる.
    \begin{align}
      1\ \r{eV} \sim 1 \times 10^4\ \si{\degreeCelsius}
    \end{align}
    である.
  \section{自然界のプラズマ}
    \begin{align}
      E = mc^2
    \end{align}
    $1\ \r{g}$の水素の核融合によるエネルギーは
    \begin{align}
      6.5 \times 10^{11} \r{J}
    \end{align}
    である.
  \section{地上に太陽を}
    \begin{itembox}[l]{ローソン条件}
      核融合反応が持続するために必要な温度,密度,閉じ込め時間の関係.  
    \end{itembox}
    \begin{itembox}[l]{ローソン図}
      ローソン条件を満たす関係をエネルギー増倍率$Q$として,
      \begin{align}
        Q = \frac{\text{核融合出力}}{\text{加熱パワー}}
      \end{align}
      で図示したもの.
    \end{itembox}
\end{document}