\documentclass{report}
\usepackage{luatexja}
\usepackage{amsmath, amssymb, type1cm, amsfonts, latexsym, mathtools, bm, amsthm, url, color}
\usepackage{multirow, hyperref, longtable, dcolumn, tablefootnote, physics}
\usepackage{tabularx, footmisc, colortbl, here, usebib, microtype}
\usepackage{graphicx, luacode, caption, fancyhdr}
\usepackage[top = 20truemm, bottom = 20truemm, left = 20truemm, right = 20truemm]{geometry}
\usepackage{ascmac, fancybox, color, tabularray, subcaption}
\usepackage{luatexja-fontspec, multicol}
\usepackage{upgreek, colortbl, mhchem}
\usepackage{biblatex, array, truthtable}
\usepackage{listings, jvlisting}
\usepackage{xcolor, siunitx, float, dcolumn}
\sisetup{
  table-format = 1.5, % 小数点以下の桁数を指定
  table-number-alignment = center, % 数値の中央揃え
}
% \abovedisplayskip = 0pt
% \belowdisplayskip = 0pt
\allowdisplaybreaks
% \DeclarePairedDelimiter{\abs}{\lvert}{\rvert}
\newcolumntype{t}{!{\vrule width 0.1pt}}
\newcolumntype{b}{!{\vrule width 1.5pt}}
\UseTblrLibrary{amsmath, booktabs, counter, diagbox, functional, hook, html, nameref, siunitx, varwidth, zref}
\setlength{\columnseprule}{0.4pt}
\captionsetup[figure]{font = bf}
\captionsetup[table]{font = bf}
\captionsetup[lstlisting]{font = bf}
\captionsetup[subfigure]{font = bf, labelformat = simple}
\setcounter{secnumdepth}{5}
\newcolumntype{d}{D{.}{.}{5}}
\newcolumntype{M}[1]{>{\centering\arraybackslash}m{#1}}
\everymath{\displaystyle}
\DeclareMathOperator*{\AND}{\cdot}
\DeclareMathOperator*{\NAND}{NAND}
% \DeclareMathOperator*{\NOT}{NOT}
\DeclareMathOperator*{\OR}{+}
% \let\oldbar\bar
\renewcommand{\i}{\mathrm{i}}
\renewcommand{\laplacian}{\Delta}
\newcommand{\NOT}[1]{\overline{#1}}
\renewcommand{\hat}[1]{\overhat{#1}}
\renewcommand{\thesubfigure}{(\alph{subfigure})}
\newcommand{\m}[3]{\multicolumn{#1}{#2}{#3}}
\renewcommand{\r}[1]{\mathrm{#1}}
\newcommand{\e}{\mathrm{e}}
\newcommand{\Ef}{E_{\mathrm{F}}}
\renewcommand{\c}{\si{\degreeCelsius}}
\renewcommand{\d}{\r{d}}
\renewcommand{\t}[1]{\texttt{#1}}
\newcommand{\kb}{k_{\mathrm{B}}}
\renewcommand{\phi}{\varphi}
% \newcommand{\dv}[3]{\frac{\d #1}{\d #2}}
% \newcommand{\pdv}[2]{\frac{\partial #1}{\partial #2}}
% \newcommand{\qtys}[#1]{\left(#1 \right)}
% \newcommand{\qtym}[#1]{\left\{#1\right\}}
% \newcommand{\qtyl}[#1]{\left[#1\right]}
\newcommand{\reff}[1]{\textbf{図\ref{#1}}}
\newcommand{\reft}[1]{\textbf{表\ref{#1}}}
\newcommand{\refe}[1]{\textbf{式\eqref{#1}}}
\newcommand{\refp}[1]{\textbf{コード\ref{#1}}}
\newcommand{\refa}[1]{\textbf{\ref{#1}}}
\renewcommand{\lstlistingname}{コード}
\renewcommand{\theequation}{\thesection.\arabic{equation}}
\renewcommand{\footrulewidth}{0.4pt}
\newcommand{\mar}[1]{\textcircled{\scriptsize #1}}
\newcommand{\combination}[2]{{}_{#1} \mathrm{C}_{#2}}
\newcommand{\thline}{\noalign{\hrule height 0.1pt}}
\newcommand{\bhline}{\noalign{\hrule height 1.5pt}}
\newcommand*{\myCurrentTime}{
  \directlua{ my_current_time() }
}
\newcommand{\Rnum}[1]{
  \ifnum #1 = 1
    I
  \fi
  \ifnum #1 = 2
    I\hspace{-1.2pt}I
  \fi
  \ifnum #1 = 3
    I\hspace{-1.2pt}I\hspace{-1.2pt}I
  \fi
  \ifnum #1 = 4
    I\hspace{-1.2pt}V
  \fi
  \ifnum #1 = 5
    V
  \fi
  \ifnum #1 = 6
    V\hspace{-1.2pt}I
  \fi
  \ifnum #1 = 7
    V\hspace{-1.2pt}I\hspace{-1.2pt}I
  \fi
  \ifnum #1 = 8
    V\hspace{-1.2pt}I\hspace{-1.2pt}I\hspace{-1.2pt}I
  \fi
  \ifnum #1 = 9
    I\hspace{-1.2pt}X
  \fi
  \ifnum #1 = 10
    X
  \fi
}
\newcommand{\cover}{
  \renewcommand{\arraystretch}{3}
  \title{物理情報工学実験報告書}
  \date{}
  \author{}
  \maketitle
  \begin{table}[H]
    \begin{flushright}
      2024年度
    \end{flushright}
    \begin{center}
      \begin{tabularx}{150mm}{|>{\centering}p{40mm}|>{\centering}p{25mm}|>{\centering}p{30mm}|>{\centering\arraybackslash}X|}
        \hline
        \Large{実験テーマ} & \multicolumn{3}{c|}{\Large{A1(直流安定化電源)}} \\ \hline
        \Large{担当教員名} & \multicolumn{3}{c|}{\Large{塚田孝祐}} \\ \hline
        \Large{実験整理番号} & \Large{002} & \Large{実験者氏名} & \Large{青木\ 陽}\\ \hline
        \Large{共同実験者氏名} & \multicolumn{3}{c|}{} \\ \hline
        \Large{曜日組} & \Large{火1班} & \Large{実験日} & \Large{6月25日} \\ \hline
        \Large{実験回} & \Large{9} & \Large{報告書提出日} & \Large{\myCurrentTime}\\ \hline
      \end{tabularx}
    \end{center}
  \end{table}
  \thispagestyle{empty} 
  \addtocounter{page}{-1}
  \clearpage
  \renewcommand{\arraystretch}{1.0}
}
\pagestyle{fancy}
\chead{物性物理II}
\rhead{}
\cfoot{\thepage}
\lhead{}
\rfoot{\t{harry\_arbrebleu}}
\setcounter{tocdepth}{4}
\makeatletter
\@addtoreset{equation}{subsection}
\makeatother
\begin{luacode*}
  function my_current_time()
    local date = os.date("*t")
    local year = date.year
    local month = date.month
    local day = date.day
    local hour = date.hour
    local min = date.min
    local sec = date.sec
    local formatted_date = string.format("%d月%d日", month, day)
    tex.sprint(formatted_date)
  end
\end{luacode*}
\lstset{
  language = Matlab, % Set the language for the code
  basicstyle = {\ttfamily},
  identifierstyle = {\small},
  commentstyle = \color{red},
  keywordstyle = \color{blue},
  ndkeywordstyle = {\small},
  stringstyle = \color{orange},
  frame={tb},
  breaklines = true,
  columns=[l]{fullflexible},
  xrightmargin = 5mm,
  xleftmargin = 5mm,
  numberstyle = {\ttfamily\scriptsize},
  stepnumber = 1,
  numbersep = 1mm,
  lineskip = -0.5ex,
  showstringspaces = false,
  numbers = left,
  frame = lines,
  backgroundcolor = \color{gray!10},
  rulecolor = \color{black!30},
}

\definecolor{mygray}{rgb}{0.5,0.5,0.5}
\definecolor{mymauve}{rgb}{0.58,0,0.82}
\definecolor{mygreen}{rgb}{0,0.6,0}

\lstset{ %
  backgroundcolor=\color{white},   % 背景色
  basicstyle=\ttfamily\footnotesize, % 基本の書体スタイル
  breakatwhitespace=false,        % 空白で行分割しない
  breaklines=true,                % 長い行は分割する
  captionpos=b,                   % キャプションの位置
  commentstyle=\color{mygreen},   % コメントのスタイル
  extendedchars=true,             % 非 ASCII 文字をサポート
  frame=single,                   % フレームの表示
  keywordstyle=\color{blue},      % キーワードのスタイル
  language=[LaTeX]TeX,            % 言語を LaTeX に設定
  numbers=left,                   % 行番号を左側に表示
  numbersep=5pt,                  % 行番号とコードの間の距離
  numberstyle=\tiny\color{mygray}, % 行番号のスタイル
  rulecolor=\color{black},        % 枠線の色
  showspaces=false,               % スペースを表示しない
  showstringspaces=false,         % 文字列内のスペースを表示しない
  showtabs=false,                 % タブを表示しない
  stepnumber=1,                   % 行番号を表示する間隔
  stringstyle=\color{mymauve},    % 文字列のスタイル
  tabsize=2,                      % タブの幅
  title=\lstname                  % タイトル
}
\lstset{
  language = C++, % Set the language for the code
  basicstyle = {\ttfamily},
  identifierstyle = {\small},
  commentstyle = \color{red},
  keywordstyle = \color{blue},
  ndkeywordstyle = {\small},
  stringstyle = \color{orange},
  frame={tb},
  breaklines = true,
  columns=[l]{fullflexible},
  xrightmargin = 5mm,
  xleftmargin = 5mm,
  numberstyle = {\ttfamily\scriptsize},
  stepnumber = 1,
  numbersep = 1mm,
  lineskip = -0.5ex,
  showstringspaces = false,
  numbers = left,
  frame = lines,
  backgroundcolor = \color{gray!10},
  rulecolor = \color{black!30},
}
\title{物性物理II}
\date{\today}
\author{\t{harry\_arbrebleu}}
\addbibresource{ref.bib}
\defbibheading{bunken}[\refname]{\section*{#1}}
\begin{document}
\section{イオン化エネルギー}
完全な復習なので式だけ示す.古典的な円運動は
\begin{align}
  \frac{m_ev^2}{r} = \frac{1}{4\pi\epsilon_0}\frac{e^2}{r^2}
\end{align}
量子化条件は
\begin{align}
  m_e vr = n\frac{h}{2\pi}
\end{align}
許される軌道半径は
\begin{align}
  r_n = n^2 \frac{h^2\epsilon_0}{\pi m_e e^2}
\end{align}
$n=1$のときの軌道半径をBohr半径と言い,$5.3\times10^{-11}$ mである.

各軌道に対する電子の全エネルギーは
\begin{align}
  E_n = -\frac{m_ee^4}{8\pi\epsilon_0^2h^2}\frac{1}{n^2}
\end{align}
$n=1$のとき,-13.6 eVである.

イオン化エネルギーの定義は
\begin{align}
  \Delta E = E_{\infty} - E_1 = 13.6\ \r{eV}
\end{align}
よって,水素原子のイオン化エネルギーは13.6 eVである.

常温・常圧では2原子あるいはそれ以上の原子が結合して分子の状態にあることが多い.この場合には,解離課程を経て原子になり,その後,電離することが多い.

\section{解離過程}
では,どのようにして電離に必要なエネルギーを与えるのだろうか.主に,電子による衝突電離と光電離がある.

\subsection{衝突電離}
イオン化エネルギー以上のエネルギーを衝突によって与えて中性粒子を電離させる方法が衝突電離である.1個の原子を電離するには5-25 eVのエネルギーが必要である.

衝突させる粒子としては荷電粒子が用いられる.なぜなら,電圧により容易に加速させられるからである.電子を使ったものを\textbf{電子衝突電離}(electron impact ionization)という.
電子はイオンに比べ,「種」になる電子が作りやすく,衝突によるエネルギーの授受の効率が良い,という特徴がある.

\subsubsection{弾性衝突と非弾性衝突}
\textbf{弾性衝突}(elastic collision)はでは,衝突前後で粒子の運動エネルギーが保存する.
\textbf{非弾性衝突}では,運動エネルギーは保存しない.しかし,全エネルギーは保存されるため,
運動エネルギーの一部が粒子の内部状態の変化に使われる.例えば,衝突電離過程や衝突励起過程に使われる.式で書くと
\begin{align}
  \frac{m_1v_1^2}{2} + \frac{m_2v_2^2}{2} = \frac{m_1v_1'^2}{2} + \frac{m_2v_2'^2}{2} + \Delta U
\end{align}
であり,$\Delta U$が粒子の内部状態変化に使われたエネルギーである.
まとめると,弾性衝突は内部状態の変化を伴わず,粒子の軌道を変えるだけである.よって,粒子拡散や熱伝導など輸送過程で重要となる.
一方,非弾性衝突では内部状態の変化を伴う.よって,プラズマ生成・消滅過程で重要となる.

\subsubsection{衝突断面積}
衝突過程を記述する物理量として\textbf{衝突断面積}(cross section of collision)がある.
これは衝突のしやすさを示すパラメータである.

例えば,半径$r_1$の粒子と半径$r_2$の粒子の衝突を考える.粒子は剛体球とみなすと,
中心間距離が$r_1+r_2$以下のときに衝突する.よって,半径$r_1+r_2$の円の面積が衝突を表すパラメータとして有用である.
したがって,衝突断面積は
\begin{align}
  \sigma = \pi(r_1 + r_2)^2
\end{align}
である.

この剛体球モデルが使えるのは,中性原子-中性原子の衝突や荷電粒子-中性原子の衝突である.これらは原子半径程度まで
接近したときに力を及ぼしあう.一方,荷電粒子-荷電粒子は,Coulomb力が遠距離まで及ぶため,剛体球として扱えない.

剛体球モデルでは,原子内の最外殻電子の軌道半径を球の半径とみなす.原子番号が大きくなるほど衝突はしやすくなる.
また,電子のde Broglie波長は
\begin{align}
  \lambda_{\r{d}} = \frac{h}{m_ev_e}
\end{align}
である.電子の速度やエネルギーが十分大きい時は電子の波動性を考慮する必要はなく,衝突断面積は原子半径程度と考えることができる.

一方,電子のエネルギーが低く,
\begin{align}
  \lambda_{\r{d}} \simeq \text{原子半径}
\end{align}
となると量子力学的な扱いが必要である.

非弾性衝突(例えば電離反応)が起こるためには,衝突する粒子はイオン化エネルギー以上の運動エネルギーを持つ必要がある.
イオン化エネルギー以下では衝突してもイオン化は起こらず,断面積は0である.よって,断面積にエネルギーの閾値があることがわかる.
また,これらの断面積は反応断面積や電離断面積と呼ばれることもある.

\subsubsection{平均自由行程}
\textbf{平均自由行程}(mean free path)は,衝突と衝突の間の平均的な飛行距離である.
$x=0$,面積$S$から,同じ速度$v$を持つ粒子Aのビームを一様に入射する.$x=0$で未衝突の粒子Aの個数を$N_{A0}$とする.
位置$x$での個数を$N_A(x)$とする.簡単のため粒子Bは止まっているとする.粒子Aと粒子Bの衝突断面積を$\sigma_{AB}$とすると,
粒子Aが$[x,x+\dd{x}]$の微小区間で粒子Bと衝突する確率は
\begin{align}
  P_{AB} = \frac{\sigma_{AB}N_B}{S}
\end{align}
である.$N_B$は$\dd{x}$に負k間れる粒子Bの個数である.
粒子Bの密度を$n_B$とすると,
\begin{align}
  N_B = n_B S \dd{x}
\end{align}
である.よって,衝突確率は
\begin{align}
  P_{AB} ~ \sigma_{AB} n_B \dd{x}
\end{align}
である.衝突の分だけビーム粒子の数は減ると考えると,
\begin{align}
  \dd{N_A} = -n_A \sigma_{AB} n_B \dd{x}
\end{align}
である.よって,
\begin{align}
  \frac{\dd{N_A}}{N_{A}} &= - \sigma_{AB} n_B \dd{x}\\
  \Rightarrow N_A(x) &= N_{A0} \exp\qty(-\frac{x}{\lambda_{AB}})
\end{align}
であり,平均自由行程は
\begin{align}
  \lambda_{AB} = \frac{1}{n_B \sigma_{AB}}
\end{align}
となる.

粒子Aが距離$x$だけ自由に飛行できる確率を求める.
$x=0$から$x$まで衝突を受けない確率は
\begin{align}
  P_A(x) = \frac{N_A(x)}{N_{A0}} = \exp\qty(-\frac{x}{\lambda_{AB}})
\end{align}
である.$[x,x+\dd{x}]$の範囲で衝突を受ける確率は
\begin{align}
  \dd{P_A(x)} = \abs{\frac{\dd{N_A}}{N_A}} = \sigma_{AB} n_B \dd{x} = \frac{\dd{x}}{\lambda_{AB}}
\end{align}
である.よって,$x$だけ自由に飛行できる確率は
\begin{align}
  P_A\times\dd{P_A} = \exp\qty(-\frac{x}{\lambda_{AB}})\frac{\dd{x}}{\lambda_{AB}}
\end{align}
である.よって,自由に飛行できる距離$x$の平均値は
\begin{align}
  \ev{x} = \int_0^{\infty} x\exp\qty(-\frac{x}{\lambda_{AB}})\frac{\dd{x}}{\lambda_{AB}} = \lambda_{AB}
\end{align}
である.$\lambda_{AB}$はまさしく平均移動距離を表している.

\subsubsection{衝突時間と衝突周波数}
衝突と衝突の間の平均的な経過時間を\textbf{衝突時間}$\tau_{AB}$という.これは平均自由行程を飛行するのに要する時間である.
\begin{align}
  \tau_{AB} = \frac{\lambda_{AB}}{v} \label{eq:collision_time}
\end{align}
で与えられる.また,
単位時間当たりの平均的な衝突回数を\textbf{衝突周波数}$\nu_{AB}$という.
\begin{align}
  \mu_{AB} = \frac{1}{\tau_{AB}} = n_B \sigma_{AB} v \label{eq:collision_frequency}
\end{align}
で与えられる.\refe{eq:collision_frequency}は粒子1個当たりの平均的な衝突回数である.粒子Aの密度を$n_A$とすると,
単位体積単位時間当たりのイオンの生成量$S$は
\begin{align}
  S_{AB} = n_A n_B \sigma_{AB} v
\end{align}
である.

\subsubsection{速度係数}
速度係数(rate coefficient)

実際には,速度分布は一様ではない.衝突周波数の速度分布に関する平均値は
\begin{align}
  \ev{v_{en}} = n_n \ev{\sigma_{en}(v)v}
\end{align}
である.ここで,
\begin{align}
  \ev{\sigma_{en}(v)} = \int_0^{\infty} \sigma_{en}(v)f(v)\dd{v}
\end{align}
を\textbf{速度係数}(反応速度係数)という.一般に,電離衝突に関わらず$\sigma \bm{v}$の速度分布にわたる平均を速度係数という.単位は$\r{m^3/s}$である.
速度係数はあるエネルギーで急激に立ち上がり,極大値を取った後に下がっていく.さがるのは衝突のエネルギーが大きすぎるとCoulomb力によるエネルギーの交換ができなくなるためである.
これを使うと,単位体積単位時間当たりの衝突回数は
\begin{align}
  S_{en} = n_e n_n \ev{\sigma_{en}(v)v}
\end{align}
と表される.

\subsubsection{衝突についての基本事項のまとめ}
衝突には弾性衝突と非弾性衝突がある.衝突断面積$\sigma$は衝突のしやすさ,電離や励起のエネルギー閾値を示す.
\begin{itembox}[l]{平均自由行程}
  \begin{align}
    \lambda_{AB} = \frac{1}{n_B \sigma_{AB}}
  \end{align}
\end{itembox}

\begin{itembox}[l]{衝突時間と衝突周波数}
  \begin{align}
    \tau_{AB} &= \frac{\lambda_{AB}}{v}\\
    \nu_{AB} &= \frac{1}{\tau_{AB}} = n_B \sigma_{AB} v
  \end{align}
\end{itembox}

\begin{itembox}[l]{単位時間単位体積当たりの衝突回数}
  \begin{align}
    S_{AB} = n_A n_B \sigma_{AB} v
  \end{align}
  粒子の速度が単一速度$v$のビームで,断面積のエネルギー依存性が無視できるとき.
\end{itembox}

\begin{itembox}[l]{速度係数}
  \begin{align}
    \ev{\sigma_{en}(v)} = \int_0^{\infty} \sigma_{en}(v)vF_e(v)\dd{v}
  \end{align}
\end{itembox}

\begin{itembox}[l]{単位時間単位体積当たりのプラズマの生成量}
  \begin{align}
    S_{en} = n_e n_n \ev{\sigma_{en}(v)v}
  \end{align}
\end{itembox}

\subsection{光電離}
光子のエネルギーがイオン化エネルギーより大きければ,電離が起こるようになる.
地球の電離層はこれによりできる.特に,太陽からの紫外線が電離を引き起こす.

光電離の条件は
\begin{align}
  A + h\nu \rightarrow A^+ + e^-
\end{align}
であるので,
\begin{align}
  h\nu \geq e V_i
\end{align}
を満たさなければならない.背片は光子のエネルギーで,右辺はイオン化エネルギーである.
整理すると,
\begin{align}
  \lambda\ \r{eV} \leq \frac{1239.8}{V_i\ [\r{eV}]}
\end{align}
となる.例えばセシウムの場合$V_i = 3.98$ eVなので,$\lambda \leq 320$ nmである.
これは紫外線の領域である.大気中では高度の高い方が短い波長,高度の低い方が長い波長の光が吸収される.

ここで,大気の構造と組成について補足する.
圧力に関する運動方程式は
\begin{align}
  \dv{p}{z} = -mgn
\end{align}
である.ここで,$p=n\kb T$なので,$T \sim \r{const.}$のとき
\begin{align}
  \dv{n}{z} = - \frac{mg}{\kb T}n
\end{align}
となる.この解は
\begin{align}
  n(z) = n_0 \exp\qty(-\frac{mgz}{\kb T}) = n_0 \exp\qty(-\frac{z}{H})
\end{align}
である.ここで,
\begin{align}
  H \equiv \frac{\kb T}{mg}
\end{align}
を\textbf{スケールハイト}(scale height)という.大気密度が$1/\e$に減衰する高度を表す.
質量が大きいほど早く減衰し,質量が小さいほど遅く減衰する.

\subsection{励起過程}
イオン化エネルギー以下でも,原子や分子の内部状態の変化を起こさせる可能性がある.これを,\textbf{励起過程}(excitation)という.

電子が高準位に励起されたとする.励起状態は一般に不安定であるため,より低い準位へと遷移し,結局は基底状態に戻る.
低準位への遷移の際,エネルギーを光として放出する.$E_m$から$E_n$への遷移に伴い放出される光は
\begin{align}
  h\nu &= E_m - E_n\\
  \Rightarrow \lambda\ \r{nm} &= \frac{1239.8}{E_m - E_n\ [\r{eV}]}
\end{align}
の波長をもつ.整理すると
\begin{align}
  \frac{1}{\lambda} = R_H\qty(\frac{1}{n^2} - \frac{1}{m^2})
\end{align}
である.$R_H$はRydberg定数で$R_H = 1.097 \times 10^7$ m$^{-1}$である.$n=1$のときをLyman系列,$n=2$のときをBalmer系列という.例えば,
$m=2\to n=1$のとき$\lambda = 121.6$ nmで真空紫外線$L_{\r{y\alpha}}$と呼ばれる.$m=3\to n=2$のとき$\lambda = 656.3$ nmで$H_{\alpha}$線と呼ばれる.
\end{document}