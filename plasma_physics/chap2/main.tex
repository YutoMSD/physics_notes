\documentclass{report}
\input{../head.tex}
\begin{document}
\section{プラズマの温度}
プラズマの\textbf{温度}は,\textbf{密度}とともにもっとも
基本的かつ重要な物理量の一つである.

壁のオンと$T$が一定の容器に.十分長い時間気体粒子を閉じ込めると
気体粒子は衝突を繰り返し熱平衡状態になる.

粒子の速度分布はMaxwell分布に従う.単位体積当たり$n$個の粒子が温度$T$で熱平衡状態に
あるとする.速度が$v_x$と$v_x + \dd{v_x}$の範囲にある粒子の個数は
\begin{align}
  f(v_x) &= \dd{v_x}\\
  f(v_x) &= A \exp\qty(-\frac{mv_x^2}{2k_B T})
\end{align}
である.$A$は規格化条件から求められる.
\begin{align}
  \int_{\infty}^{\infty} f(v_x) \dd{v_x} &= n\\
  \Rightarrow A &= n\sqrt{\frac{m}{2\pi k_B T}}
\end{align}
温度は速度分布の広がりを表すパラメータである.高温ほど,速度が大きい粒子が多い.

粒子の平均エネルギーを計算する.
\begin{align}
  E_{av} = \frac{1}{n}\int_{\infty}^{\infty} \frac{1}{2}mv_x^2f(v_x) \dd{v_x} = \frac{1}{2}k_B T
\end{align}

熱速度
\begin{align}
  v_{th} = \sqrt{\frac{2k_B T}{m}}
\end{align}
とすると,
\begin{align}
  f(\pm v_{th}) = \frac{f(0)}{\e}
\end{align}
である.

3次元に拡張する.
\begin{align}
  f(v_x,v_y,v_z) &= n\qty(\frac{m}{2\pi k_B T})^{3/2} \exp\qty[-\frac{1}{2}\frac{m(v_x^2 + v_y^2 + v_z^2)}{k_B T}]\\
  E_{av} &= \frac{3}{2}k_B T
\end{align}

速さ$v$に関する速度分布は
\begin{align}
  F(v) = 4\pi n \qty(\frac{m}{2\pi k_B T})^{3/2} v^2 \exp\qty[-\frac{1}{2}\frac{mv^2}{k_B T}]
\end{align}
である.これは$v=v_{th}$で最大となる.
\end{document}