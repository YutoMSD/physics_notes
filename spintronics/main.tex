\documentclass{ltjarticle}
\input{head.tex}
\title{Basic ideas of Spintronics and Orbitronics}
\author{Yuto Masuda}
\date{\today}
\begin{document}
\maketitle

\tableofcontents

\textbf{研究室HPを見て内容がわかるようになることを目指す}.最初の方は物質中の軌道角運動量を無視する.
\section{Spin Current}
\subsection{Conservation of Angular Momentum and Spin current}
電荷保存則
\begin{align}
  \frac{\partial}{\partial t}\rho = -\nabla \cdot \bm{j}_{\r{c}}
\end{align}
角運動量保存則
\begin{align}
  \frac{\partial}{\partial t}\bm{S} = -\nabla\cdot\bm{j}_{\r{s}}
\end{align}
$\bm{M} = \gamma\bm{S}$だから,磁化の時間変化はスピン角運動量の流れを生む.

\subsection{Spin Relaxation}
\begin{align}
  \frac{\partial}{\partial t}\bm{S} = -\nabla\cdot\bm{j}_{\r{s}} + \bm{T}
\end{align}
スピン流の伝搬距離は数 nmほど.緩和機構には
\begin{enumerate}
  \item EY機構
  \item D'yakonov-Perel'機構
  \item スピンメモリーロス
\end{enumerate}
などがある.
\begin{align}
  \psi \sim a\e^{\i\epsilon_{\uparrow}t/\hbar}\ket{\uparrow} + b\e^{\i\epsilon_{\downarrow}t/\hbar}\ket{b}
\end{align}
\subsection{Cunduction-electron Spin Current}
スピン偏極した伝導電子の流れ.強磁性体

\section{Spin Dynamics in Solids}
\subsection{LLG equation}
\begin{align}
  \dv{\bm{M}}{t} &= -\frac{\i}{\hbar}\qty[\bm{M},\hat{H}]\\
  \hat{H} &= -\bm{M}\cdot\bm{H}_{\r{eff}}\\
  \Rightarrow \dv{\bm{M}}{t} &= -\gamma \bm{M}\times\bm{H}_{\r{eff}} \label{landau-eq}
\end{align}
有効磁場$\bm{H}_{\r{eff}}$形状磁気異方性,結晶磁気異方性,界面磁気異方性による効果を含む.
\refe{landau-eq}は歳差運動を表している.これでは歳差運動し続けるので緩和項を加える.
\begin{align}
  \frac{\r{d}}{\dd{t}}\bm{M} = -\gamma \bm{M}\times \bm{H}_{\r{eff}} + \frac{\alpha}{M}\bm{M}\times \frac{\r{d}}{\dd{t}}\bm{M} \label{LLG-eq}
\end{align}
\refe{LLG-eq}をLLG方程式という.
\subsection{Exchange Spin Current}
Heisenbergの交換相互作用より,
\begin{align}
  \hat{H} = -2J\sum_{i,j} \bm{S}_i \cdot \bm{S}_j
\end{align}
である.有効磁場は
\begin{align}
  \bm{H}_{\r{eff}} = -\frac{\delta E(\bm{m})}{\delta\bm{m}} = -2\frac{J}{\gamma}\sum_{i,j} \bm{S}_j
\end{align}
であるので,
\begin{align}
  \dv{\bm{M}}{t} = 2J\bm{M}_i \times \sum_{j} \bm{S}_j + \bm{T}'
\end{align}
が磁化の運動方程式.$\bm{T}'$は緩和項.連続体近似$\bm{S}_i = \bm{S}(\bm{r}),\ \bm{S}_j = \bm{S}(\bm{r+a})$を使う.
\begin{align}
  \bm{S}(\bm{r}+\bm{a}) = \bm{S}(\bm{r}) + \frac{\partial\bm{S}}{\bm{r}}\cdot\bm{a} + \frac{1}{2}\frac{\partial^2\bm{S}}{\partial\bm{r}^2}\bm{a}^2 + \cdots
\end{align}
\begin{align}
  \sum_{j} \bm{S}_j \to \frac{\partial^2 \bm{S}}{\partial \bm{r}^2} a^2 = \nabla^2 \bm{M}a^2
\end{align}
\begin{align}
  \frac{\r{d}}{\dd{t}}\bm{M}(\bm{r}) = \frac{2Ja^2}{\gamma}\bm{M}(\bm{r})\times\nabla^2\bm{M}(\bm{r}) + \bm{T}' = -A\gamma\nabla\cdot(\bm{M}(\bm{r})\times\nabla\bm{M}(\bm{r})) + \bm{T}'
\end{align}
この$\bm{j}_{\r{s}} = A\bm{M}(\bm{r})\times\nabla\bm{M}(\bm{r})$を交換スピン流という.磁気モーメントの勾配があるとき,それが消えるまで磁化を変化させる作用が働く.
交換相互作用が運ぶ角運動量の流れ.

\subsection{Spin-wave Spin Current and Magnon}
磁化は磁場を中心に歳差運動する.強磁性体内では交換相互作用のために歳差運動が伝わり,全体で波ができる.
これをスピン波といい,それを量子化したものがマグノン.マグノンはスピン流を運ぶ.マグノンによる伝搬は電子伝導を伴わないため
絶縁体内でも可能である.
\subsection{Spin Torque}
スピン偏極した電流が強磁性体に注入されたとき.電流のスピンの向きは強磁性体の磁化の向きにそろおうとする.このとき,
角運動量保存則により強磁性体の磁化は逆向きのトルクを受ける.
\begin{align}
  \bm{S}_1 \times (\bm{S}_1\times \bm{S}_2)
\end{align}
これをスピン移行トルク(STT)という.STTは磁化と垂直な向きに働く.磁化の大きさを変えるには大きなエネルギーが必要なため,
磁化と平行向きには働かない.
\subsection{Spin Pumping}
スピントルクの逆現象がスピンポンピング.強磁性中の磁化を時間変化させることでスピン流を生み出す.
\subsubsection{スピン蓄積}
\begin{align}
  \bm{j}_s = \bm{j}_{\uparrow} - \bm{j}_{\downarrow} = \frac{1}{e}\nabla(\sigma_{\uparrow}\mu_{\uparrow} - \sigma_{\downarrow}\mu_{\downarrow}) = \frac{\sigma}{2e}\nabla\mu_{s}
\end{align}
スピン蓄積$\mu_s=\mu_{\uparrow}- \mu_{\downarrow}$
\begin{align}
  \nabla^2\mu_s=\frac{1}{\lambda^2}\mu_s
\end{align}
スピン蓄積があるとスピン流が流れる.
\subsubsection{スピンポンピング}
スピン蓄積
\begin{align}
  \Delta N \simeq -\frac{\chi}{\gamma}\frac{\Gamma^2 + 1}{\Gamma}\qty(\frac{1}{\Gamma}\bm{m}\times\dot{\bm{m}} + \dot{\bm{m}})
\end{align}
\section{How to induce Spin Current?}
スピン流の生成方法を紹介する.
\subsection{Spin Hall effect}
電流と直行する向きにスピン流が発生.磁場を必要としない.強いスピン軌道相互作用が必要.
外因性と内因性がある.外因性は不純物による散乱.内因性はベリー曲率(バンド構造に依る)
\begin{align}
  \dot{\bm{r}} &= \frac{1}{\hbar}\dv{E}{\bm{k}} - \dot{\bm{k}} \times \bm{\Omega}_n\\
  \hbar\dot{\bm{k}} &= -e\bm{E} - e\dot{\bm{r}}\times \mu_0 \bm{H}
\end{align}
\subsection{Spin Rashba effect}
空間反転対称性の破れた系
\begin{align}
  H = \bm{s}\cdot(\bm{p}\times\bm{e}_z)
\end{align}
\begin{align}
  H = 
  \begin{pmatrix}
    \hbar^2(k_x^2 + k_y^2)/2m & \alpha_{\r{SR}}(\i k_x + k_y)\\
    \alpha_{\r{SR}}(-\i k_x + k_y) & \hbar(k_x^2 + k_y^2)/2m
  \end{pmatrix}
\end{align}
運動量とスピンがロッキング.
\subsection{Light-induced Spin Current}
円偏光のスピン.
\subsection{Spin Seebeck effect}
熱勾配をかける.スピンの歳差運動は反時計回りしかありえない.
\section{How to measure Spin Current?}
\section{Orbital Current}
今まで,物質中の軌道角運動量は無視されてきた.
空間反転対称性から
\begin{align}
  \langle \bm{L} \rangle_{\bm{k}} = \langle \bm{L} \rangle_{-\bm{k}}
\end{align}
時間反転対称性から
\begin{align}
  \langle \bm{L} \rangle_{\bm{k}} = -\langle \bm{L} \rangle_{-\bm{k}}
\end{align}
\section{Orbitronics}
軌道Hall効果.軌道Rashba効果.軌道トルク.軌道ポンピング.
\section*{Reference}
\begin{enumerate}
  \item スピン流とトポロジカル絶縁体,ざっくり書いてる.
  \item Spin Current,いっぱい書いてる
  \item 
\end{enumerate}
\end{document}