\documentclass{ltjarticle}
\usepackage{luatexja}
\usepackage{amsmath, amssymb, type1cm, amsfonts, latexsym, mathtools, bm, amsthm, url, color}
\usepackage{multirow, hyperref, longtable, dcolumn, tablefootnote, physics}
\usepackage{tabularx, footmisc, colortbl, here, usebib, microtype}
\usepackage{graphicx, luacode, caption, fancyhdr}
\usepackage[top = 20truemm, bottom = 20truemm, left = 20truemm, right = 20truemm]{geometry}
\usepackage{ascmac, fancybox, color, tabularray, subcaption}
\usepackage{luatexja-fontspec, multicol}
\usepackage{upgreek, colortbl, mhchem}
\usepackage{biblatex, array, truthtable}
\usepackage{listings, jvlisting}
\usepackage{xcolor, siunitx, float, dcolumn}
\sisetup{
  table-format = 1.5, % 小数点以下の桁数を指定
  table-number-alignment = center, % 数値の中央揃え
}
% \abovedisplayskip = 0pt
% \belowdisplayskip = 0pt
\allowdisplaybreaks
% \DeclarePairedDelimiter{\abs}{\lvert}{\rvert}
\newcolumntype{t}{!{\vrule width 0.1pt}}
\newcolumntype{b}{!{\vrule width 1.5pt}}
\UseTblrLibrary{amsmath, booktabs, counter, diagbox, functional, hook, html, nameref, siunitx, varwidth, zref}
\setlength{\columnseprule}{0.4pt}
\captionsetup[figure]{font = bf}
\captionsetup[table]{font = bf}
\captionsetup[lstlisting]{font = bf}
\captionsetup[subfigure]{font = bf, labelformat = simple}
\setcounter{secnumdepth}{5}
\newcolumntype{d}{D{.}{.}{5}}
\newcolumntype{M}[1]{>{\centering\arraybackslash}m{#1}}
\everymath{\displaystyle}
\DeclareMathOperator*{\AND}{\cdot}
\DeclareMathOperator*{\NAND}{NAND}
% \DeclareMathOperator*{\NOT}{NOT}
\DeclareMathOperator*{\OR}{+}
% \let\oldbar\bar
\renewcommand{\i}{\mathrm{i}}
\renewcommand{\laplacian}{\Delta}
\newcommand{\NOT}[1]{\overline{#1}}
\renewcommand{\hat}[1]{\overhat{#1}}
\renewcommand{\thesubfigure}{(\alph{subfigure})}
\newcommand{\m}[3]{\multicolumn{#1}{#2}{#3}}
\renewcommand{\r}[1]{\mathrm{#1}}
\newcommand{\e}{\mathrm{e}}
\newcommand{\Ef}{E_{\mathrm{F}}}
\renewcommand{\c}{\si{\degreeCelsius}}
\renewcommand{\d}{\r{d}}
\renewcommand{\t}[1]{\texttt{#1}}
\newcommand{\kb}{k_{\mathrm{B}}}
\renewcommand{\phi}{\varphi}
% \newcommand{\dv}[3]{\frac{\d #1}{\d #2}}
% \newcommand{\pdv}[2]{\frac{\partial #1}{\partial #2}}
% \newcommand{\qtys}[#1]{\left(#1 \right)}
% \newcommand{\qtym}[#1]{\left\{#1\right\}}
% \newcommand{\qtyl}[#1]{\left[#1\right]}
\newcommand{\reff}[1]{\textbf{図\ref{#1}}}
\newcommand{\reft}[1]{\textbf{表\ref{#1}}}
\newcommand{\refe}[1]{\textbf{式\eqref{#1}}}
\newcommand{\refp}[1]{\textbf{コード\ref{#1}}}
\newcommand{\refa}[1]{\textbf{\ref{#1}}}
\renewcommand{\lstlistingname}{コード}
\renewcommand{\theequation}{\thesection.\arabic{equation}}
\renewcommand{\footrulewidth}{0.4pt}
\newcommand{\mar}[1]{\textcircled{\scriptsize #1}}
\newcommand{\combination}[2]{{}_{#1} \mathrm{C}_{#2}}
\newcommand{\thline}{\noalign{\hrule height 0.1pt}}
\newcommand{\bhline}{\noalign{\hrule height 1.5pt}}
\newcommand*{\myCurrentTime}{
  \directlua{ my_current_time() }
}
\newcommand{\Rnum}[1]{
  \ifnum #1 = 1
    I
  \fi
  \ifnum #1 = 2
    I\hspace{-1.2pt}I
  \fi
  \ifnum #1 = 3
    I\hspace{-1.2pt}I\hspace{-1.2pt}I
  \fi
  \ifnum #1 = 4
    I\hspace{-1.2pt}V
  \fi
  \ifnum #1 = 5
    V
  \fi
  \ifnum #1 = 6
    V\hspace{-1.2pt}I
  \fi
  \ifnum #1 = 7
    V\hspace{-1.2pt}I\hspace{-1.2pt}I
  \fi
  \ifnum #1 = 8
    V\hspace{-1.2pt}I\hspace{-1.2pt}I\hspace{-1.2pt}I
  \fi
  \ifnum #1 = 9
    I\hspace{-1.2pt}X
  \fi
  \ifnum #1 = 10
    X
  \fi
}
\newcommand{\cover}{
  \renewcommand{\arraystretch}{3}
  \title{物理情報工学実験報告書}
  \date{}
  \author{}
  \maketitle
  \begin{table}[H]
    \begin{flushright}
      2024年度
    \end{flushright}
    \begin{center}
      \begin{tabularx}{150mm}{|>{\centering}p{40mm}|>{\centering}p{25mm}|>{\centering}p{30mm}|>{\centering\arraybackslash}X|}
        \hline
        \Large{実験テーマ} & \multicolumn{3}{c|}{\Large{A1(直流安定化電源)}} \\ \hline
        \Large{担当教員名} & \multicolumn{3}{c|}{\Large{塚田孝祐}} \\ \hline
        \Large{実験整理番号} & \Large{002} & \Large{実験者氏名} & \Large{青木\ 陽}\\ \hline
        \Large{共同実験者氏名} & \multicolumn{3}{c|}{} \\ \hline
        \Large{曜日組} & \Large{火1班} & \Large{実験日} & \Large{6月25日} \\ \hline
        \Large{実験回} & \Large{9} & \Large{報告書提出日} & \Large{\myCurrentTime}\\ \hline
      \end{tabularx}
    \end{center}
  \end{table}
  \thispagestyle{empty} 
  \addtocounter{page}{-1}
  \clearpage
  \renewcommand{\arraystretch}{1.0}
}
\pagestyle{fancy}
\chead{物性物理II}
\rhead{}
\cfoot{\thepage}
\lhead{}
\rfoot{\t{harry\_arbrebleu}}
\setcounter{tocdepth}{4}
\makeatletter
\@addtoreset{equation}{subsection}
\makeatother
\begin{luacode*}
  function my_current_time()
    local date = os.date("*t")
    local year = date.year
    local month = date.month
    local day = date.day
    local hour = date.hour
    local min = date.min
    local sec = date.sec
    local formatted_date = string.format("%d月%d日", month, day)
    tex.sprint(formatted_date)
  end
\end{luacode*}
\lstset{
  language = Matlab, % Set the language for the code
  basicstyle = {\ttfamily},
  identifierstyle = {\small},
  commentstyle = \color{red},
  keywordstyle = \color{blue},
  ndkeywordstyle = {\small},
  stringstyle = \color{orange},
  frame={tb},
  breaklines = true,
  columns=[l]{fullflexible},
  xrightmargin = 5mm,
  xleftmargin = 5mm,
  numberstyle = {\ttfamily\scriptsize},
  stepnumber = 1,
  numbersep = 1mm,
  lineskip = -0.5ex,
  showstringspaces = false,
  numbers = left,
  frame = lines,
  backgroundcolor = \color{gray!10},
  rulecolor = \color{black!30},
}

\definecolor{mygray}{rgb}{0.5,0.5,0.5}
\definecolor{mymauve}{rgb}{0.58,0,0.82}
\definecolor{mygreen}{rgb}{0,0.6,0}

\lstset{ %
  backgroundcolor=\color{white},   % 背景色
  basicstyle=\ttfamily\footnotesize, % 基本の書体スタイル
  breakatwhitespace=false,        % 空白で行分割しない
  breaklines=true,                % 長い行は分割する
  captionpos=b,                   % キャプションの位置
  commentstyle=\color{mygreen},   % コメントのスタイル
  extendedchars=true,             % 非 ASCII 文字をサポート
  frame=single,                   % フレームの表示
  keywordstyle=\color{blue},      % キーワードのスタイル
  language=[LaTeX]TeX,            % 言語を LaTeX に設定
  numbers=left,                   % 行番号を左側に表示
  numbersep=5pt,                  % 行番号とコードの間の距離
  numberstyle=\tiny\color{mygray}, % 行番号のスタイル
  rulecolor=\color{black},        % 枠線の色
  showspaces=false,               % スペースを表示しない
  showstringspaces=false,         % 文字列内のスペースを表示しない
  showtabs=false,                 % タブを表示しない
  stepnumber=1,                   % 行番号を表示する間隔
  stringstyle=\color{mymauve},    % 文字列のスタイル
  tabsize=2,                      % タブの幅
  title=\lstname                  % タイトル
}
\lstset{
  language = C++, % Set the language for the code
  basicstyle = {\ttfamily},
  identifierstyle = {\small},
  commentstyle = \color{red},
  keywordstyle = \color{blue},
  ndkeywordstyle = {\small},
  stringstyle = \color{orange},
  frame={tb},
  breaklines = true,
  columns=[l]{fullflexible},
  xrightmargin = 5mm,
  xleftmargin = 5mm,
  numberstyle = {\ttfamily\scriptsize},
  stepnumber = 1,
  numbersep = 1mm,
  lineskip = -0.5ex,
  showstringspaces = false,
  numbers = left,
  frame = lines,
  backgroundcolor = \color{gray!10},
  rulecolor = \color{black!30},
}
\title{物性物理II}
\date{\today}
\author{\t{harry\_arbrebleu}}
\addbibresource{ref.bib}
\defbibheading{bunken}[\refname]{\section*{#1}}
\title{Basic ideas of Spintronics and Orbitronics}
\author{Yuto Masuda}
\date{\today}
\begin{document}
\maketitle

\tableofcontents

\textbf{研究室HPを見て内容がわかるようになることを目指す}.最初の方は物質中の軌道角運動量を無視する.
\section{Spin Current}
\subsection{Conservation of Angular Momentum and Spin current}
電荷保存則
\begin{align}
  \frac{\partial}{\partial t}\rho = -\nabla \cdot \bm{j}_{\r{c}}
\end{align}
角運動量保存則
\begin{align}
  \frac{\partial}{\partial t}\bm{S} = -\nabla\cdot\bm{j}_{\r{s}}
\end{align}
$\bm{M} = \gamma\bm{S}$だから,磁化の時間変化はスピン角運動量の流れを生む.

\subsection{Spin Relaxation}
\begin{align}
  \frac{\partial}{\partial t}\bm{S} = -\nabla\cdot\bm{j}_{\r{s}} + \bm{T}
\end{align}
スピン流の伝搬距離は数 nmほど.緩和機構には
\begin{enumerate}
  \item EY機構
  \item D'yakonov-Perel'機構
  \item スピンメモリーロス
\end{enumerate}
などがある.
\begin{align}
  \psi \sim a\e^{\i\epsilon_{\uparrow}t/\hbar}\ket{\uparrow} + b\e^{\i\epsilon_{\downarrow}t/\hbar}\ket{b}
\end{align}
\subsection{Cunduction-electron Spin Current}
スピン偏極した伝導電子の流れ.強磁性体

\section{Spin Dynamics in Solids}
\subsection{LLG equation}
\begin{align}
  \dv{\bm{M}}{t} &= -\frac{\i}{\hbar}\qty[\bm{M},\hat{H}]\\
  \hat{H} &= -\bm{M}\cdot\bm{H}_{\r{eff}}\\
  \Rightarrow \dv{\bm{M}}{t} &= -\gamma \bm{M}\times\bm{H}_{\r{eff}} \label{landau-eq}
\end{align}
有効磁場$\bm{H}_{\r{eff}}$形状磁気異方性,結晶磁気異方性,界面磁気異方性による効果を含む.
\refe{landau-eq}は歳差運動を表している.これでは歳差運動し続けるので緩和項を加える.
\begin{align}
  \frac{\r{d}}{\dd{t}}\bm{M} = -\gamma \bm{M}\times \bm{H}_{\r{eff}} + \frac{\alpha}{M}\bm{M}\times \frac{\r{d}}{\dd{t}}\bm{M} \label{LLG-eq}
\end{align}
\refe{LLG-eq}をLLG方程式という.
\subsection{Exchange Spin Current}
Heisenbergの交換相互作用より,
\begin{align}
  \hat{H} = -2J\sum_{i,j} \bm{S}_i \cdot \bm{S}_j
\end{align}
である.有効磁場は
\begin{align}
  \bm{H}_{\r{eff}} = -\frac{\delta E(\bm{m})}{\delta\bm{m}} = -2\frac{J}{\gamma}\sum_{i,j} \bm{S}_j
\end{align}
であるので,
\begin{align}
  \dv{\bm{M}}{t} = 2J\bm{M}_i \times \sum_{j} \bm{S}_j + \bm{T}'
\end{align}
が磁化の運動方程式.$\bm{T}'$は緩和項.連続体近似$\bm{S}_i = \bm{S}(\bm{r}),\ \bm{S}_j = \bm{S}(\bm{r+a})$を使う.
\begin{align}
  \bm{S}(\bm{r}+\bm{a}) = \bm{S}(\bm{r}) + \frac{\partial\bm{S}}{\bm{r}}\cdot\bm{a} + \frac{1}{2}\frac{\partial^2\bm{S}}{\partial\bm{r}^2}\bm{a}^2 + \cdots
\end{align}
\begin{align}
  \sum_{j} \bm{S}_j \to \frac{\partial^2 \bm{S}}{\partial \bm{r}^2} a^2 = \nabla^2 \bm{M}a^2
\end{align}
\begin{align}
  \frac{\r{d}}{\dd{t}}\bm{M}(\bm{r}) = \frac{2Ja^2}{\gamma}\bm{M}(\bm{r})\times\nabla^2\bm{M}(\bm{r}) + \bm{T}' = -A\gamma\nabla\cdot(\bm{M}(\bm{r})\times\nabla\bm{M}(\bm{r})) + \bm{T}'
\end{align}
この$\bm{j}_{\r{s}} = A\bm{M}(\bm{r})\times\nabla\bm{M}(\bm{r})$を交換スピン流という.磁気モーメントの勾配があるとき,それが消えるまで磁化を変化させる作用が働く.
交換相互作用が運ぶ角運動量の流れ.

\subsection{Spin-wave Spin Current and Magnon}
磁化は磁場を中心に歳差運動する.強磁性体内では交換相互作用のために歳差運動が伝わり,全体で波ができる.
これをスピン波といい,それを量子化したものがマグノン.マグノンはスピン流を運ぶ.マグノンによる伝搬は電子伝導を伴わないため
絶縁体内でも可能である.
\subsection{Spin Torque}
スピン偏極した電流が強磁性体に注入されたとき.電流のスピンの向きは強磁性体の磁化の向きにそろおうとする.このとき,
角運動量保存則により強磁性体の磁化は逆向きのトルクを受ける.
\begin{align}
  \bm{S}_1 \times (\bm{S}_1\times \bm{S}_2)
\end{align}
これをスピン移行トルク(STT)という.STTは磁化と垂直な向きに働く.磁化の大きさを変えるには大きなエネルギーが必要なため,
磁化と平行向きには働かない.
\subsection{Spin Pumping}
スピントルクの逆現象がスピンポンピング.強磁性中の磁化を時間変化させることでスピン流を生み出す.
\subsubsection{スピン蓄積}
\begin{align}
  \bm{j}_s = \bm{j}_{\uparrow} - \bm{j}_{\downarrow} = \frac{1}{e}\nabla(\sigma_{\uparrow}\mu_{\uparrow} - \sigma_{\downarrow}\mu_{\downarrow}) = \frac{\sigma}{2e}\nabla\mu_{s}
\end{align}
スピン蓄積$\mu_s=\mu_{\uparrow}- \mu_{\downarrow}$
\begin{align}
  \nabla^2\mu_s=\frac{1}{\lambda^2}\mu_s
\end{align}
スピン蓄積があるとスピン流が流れる.
\subsubsection{スピンポンピング}
スピン蓄積
\begin{align}
  \Delta N \simeq -\frac{\chi}{\gamma}\frac{\Gamma^2 + 1}{\Gamma}\qty(\frac{1}{\Gamma}\bm{m}\times\dot{\bm{m}} + \dot{\bm{m}})
\end{align}
\section{How to induce Spin Current?}
スピン流の生成方法を紹介する.
\subsection{Spin Hall effect}
電流と直行する向きにスピン流が発生.磁場を必要としない.強いスピン軌道相互作用が必要.
外因性と内因性がある.外因性は不純物による散乱.内因性はベリー曲率(バンド構造に依る)
\begin{align}
  \dot{\bm{r}} &= \frac{1}{\hbar}\dv{E}{\bm{k}} - \dot{\bm{k}} \times \bm{\Omega}_n\\
  \hbar\dot{\bm{k}} &= -e\bm{E} - e\dot{\bm{r}}\times \mu_0 \bm{H}
\end{align}
\subsection{Spin Rashba effect}
空間反転対称性の破れた系
\begin{align}
  H = \bm{s}\cdot(\bm{p}\times\bm{e}_z)
\end{align}
\begin{align}
  H = 
  \begin{pmatrix}
    \hbar^2(k_x^2 + k_y^2)/2m & \alpha_{\r{SR}}(\i k_x + k_y)\\
    \alpha_{\r{SR}}(-\i k_x + k_y) & \hbar(k_x^2 + k_y^2)/2m
  \end{pmatrix}
\end{align}
運動量とスピンがロッキング.
\subsection{Light-induced Spin Current}
円偏光のスピン.
\subsection{Spin Seebeck effect}
熱勾配をかける.スピンの歳差運動は反時計回りしかありえない.
\section{How to measure Spin Current?}
\section{Orbital Current}
今まで,物質中の軌道角運動量は無視されてきた.
空間反転対称性から
\begin{align}
  \langle \bm{L} \rangle_{\bm{k}} = \langle \bm{L} \rangle_{-\bm{k}}
\end{align}
時間反転対称性から
\begin{align}
  \langle \bm{L} \rangle_{\bm{k}} = -\langle \bm{L} \rangle_{-\bm{k}}
\end{align}
\section{Orbitronics}
軌道Hall効果.軌道Rashba効果.軌道トルク.軌道ポンピング.
\section*{Reference}
\begin{enumerate}
  \item スピン流とトポロジカル絶縁体,ざっくり書いてる.
  \item Spin Current,いっぱい書いてる
  \item 
\end{enumerate}
\end{document}